\begin{frame}{Artigo Científico}
    \begin{block}{Os Autores}
        \begin{itemize}
            \item normalmente, a ordem dos nomes no artigo inicia pelo autor principal
            \item recomenda-se que coloque os nomes apenas dos que realmente contribuíram para o trabalho
        \end{itemize}
    \end{block}
\end{frame}

\begin{frame}{Artigo Científico}
    \begin{block}{Motivação para escrever}
        \begin{itemize}
            \item é preciso saber claramente sobre o que vai se escrever
            \item comunica uma ideia em um espaço muito curto, não deve desperdiçar linhas
            \item apresenta uma pesquisa e não deve ser usado para apresentar ferramentas, protótipos ou processos
        \end{itemize}
    \end{block}
\end{frame}

\begin{frame}{Artigo Científico}
    \begin{block}{Motivação para escrever}
        \begin{itemize}
            \item para ajudar o autor, vale se questionar alguns pontos
            \item ``O que o meu artigo está tentando comunicar?'' e ``Qual é o público-alvo do meu artigo?''
            \item se os autores não tiverem respostas objetivas para estas questões, pode ser necessário reavaliar o trabalho.
        \end{itemize}
    \end{block}
\end{frame}

\begin{frame}{Artigo Científico}
    \begin{block}{Trabalhos correlatos}
        \begin{itemize}
            \item deve mencionar outros trabalhos que são correlatos ao tema pesquisado
            \item mas e quando, de fato, não houver trabalho correlato para mencionar?
            \begin{itemize}
                 \item delimitar claramente o escopo da pesquisa bibliográfica, especificando intervalo de anos pesquisados (10, 20 anos, etc)
                 \item pesquisar nas melhores bases de dados de períodicos e conferências da área
                 \item criar mais de uma forma relacionar artigos encontrados, como ``não relacionados'', ``moderadamente relacionados''
            \end{itemize}
        \end{itemize}
    \end{block}
\end{frame}

\begin{frame}{Artigo Científico}
    \begin{block}{Contribuição do artigo}
        \begin{itemize}
            \item não exagere a contribuição, por outro lado, não seja modesto demais
            \item quem estiver avaliando o artigo precisa ser convencido de sua importância
            \item com isso, deve-se apresentar provas, evidências, descrições
            \item os resultados devem ser apresentados já no resumo do trabalho e no corpo textual, a demonstração de como se chegou lá
        \end{itemize}
    \end{block}
\end{frame}

\begin{frame}{Artigo Científico}
    \begin{block}{Tipos de artigos}
        \begin{itemize}
            \item Artigo Teórico: apresenta um conjunto de definições e posteriormente passa a provar propriedades lógicas desse conjunto
            \item Relato de Experiência: conta uma história informativa sobre um experimento e suas observações
            \item Artigo sobre Métodos: é uma apresentação de um método e suas vantagens sobre outros anteriormente propostos (computação)
        \end{itemize}
    \end{block}
\end{frame}

\begin{frame}{Artigo Científico}
    \begin{block}{Veículos de publicação}
        \begin{itemize}
            \item Periódicos: são considerados a publicação mais importante por todas as áreas da ciência
            \item Conferências: não são tão valorizadas quanto os periódicos por outras áreas mas têm peso na área de Computação
            \item Workshops e seminários: eventos-satélites de conferências maiores
            \item Livros e capítulos de livros: muito valorizados, são publicações que, em geral, o objetivo é ser didático para compreensão por parte de um público amplo
        \end{itemize}
    \end{block}
\end{frame}

\begin{frame}{Artigo Científico}
    \begin{block}{Qualis}
        \begin{itemize}
            \item ** não é uma base de indexação de periódicos **
            \item só existe como ferramenta para a avaliação de programas de pós-graduação stricto sensu
            \item cada periódico passa a ter um conceito único na lista do Qualis
            \begin{itemize}
                \item são considerados os veículos com corpo editorial reconhecido
                \item avaliação pelos pares
                \item com ISSN
                \item que constem em bases de dados de periódicos reconhecidos internacionalmente
            \end{itemize}
        \end{itemize}
    \end{block}
\end{frame}

\begin{frame}{Artigo Científico}
    \begin{block}{Qualis}
        \begin{itemize}
            \item os periódicos são classificados em níveis:  A1, A2, A3, A4, B1, B2, B3, B4 e C
            \item o nível mais elevado é A1, periódico de excelência
            \item o nível mais baixo, C, é considerado como não atendendo a boas práticas editoriais
        \end{itemize}
    \end{block}
\end{frame}

%\begin{frame}{Resultados}
%\end{frame}
