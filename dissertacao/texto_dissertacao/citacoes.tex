\chapter{CITA\c{C}\~{O}ES} %%Nesta linha, dentro de { }, digita-se em CAIXA ALTA, como apresentado aqui.

As citações são informa\c{c}\~{o}es extra\'idas de fonte consultada pelo autor da obra em desenvolvimento. Podem ser diretas, indiretas ou citação de citação. Para exemplos, consultar o apêncice D no Manual de Normalização de Trabalhos Acadêmicos disponível no \textit{link} abaixo: \\ 
\url{https://www2.ufjf.br/biblioteca/wp-content/uploads/sites/56/2020/08/Manual-2020-revisado.pdf}

\section{SISTEMA AUTOR-DATA} %%Nesta linha, dentro de { }, digita-se o nome da se\c{c}\~ao secund\'aria em CAIXA ALTA, como apresentado aqui.

Para o sistema autor-data, considere: 
\begin{itemize}
 \item[a)] \textbf{citação direta} \'e caracterizada pela transcri\c{c}\~{a}o textual da parte consultada. Se com at\'e tr\^es linhas, deve estar entre aspas duplas, exatamente como na obra consultada. Se com mais de tr\^es linhas, devem estar com recuo de 4 cm da margem esquerda, com letra menor (um ponto), espaçamento simples, sem aspas. Sendo a chamada: (AUTOR, data e p\'agina) ou na senten\c{c}a Autor (data, p\'agina). 
 \item[b)] \textbf{cita\c{c}\~{a}o indireta} \'e aquela em que o texto foi baseado na(s) obra(s) consultada(s). Em caso de mais de tr\^es fontes consultadas, a cita\c{c}\~{a}o deve seguir a ordem alfab\'etica. 
 \item[c)] \textbf{A cita\c{c}\~{a}o de cita\c{c}\~{a}o} \'e baseada em um texto em que n\~ao houve acesso ao original. 
\end{itemize} 
 
\section{SISTEMA NUM\'ERICO} %%Nesta linha, dentro de { }, digita-se o nome da se\c{c}\~ao secund\'aria em CAIXA ALTA, como apresentado aqui.

\textbf{Para o sistema num\'erico:} 

A indica\c{c}\~{a}o da fonte \'e feita por uma numera\c{c}\~{a}o \'unica e consecutiva respeitando a ordem que aparece no texto. Deve-se usar algarismos ar\'abicos remetendo \'a lista de refer\^encias. A indica\c{c}\~{a}o da numera\c{c}\~{a}o \'e apresentada entre par\^enteses no corpo do texto ou como expoente. N\~ao usar colchetes. O autor pode aparecer ou n\~ao no texto. Para separar diversos autores, utiliza-se v\'irgula. N\~{a}o utilizar nota explicativa (rodap\'{e}) quando utilizar o sistema num\'{e}rico. 
Observe os exemplos no Manual de Normaliza\c{c}\~{a}o de Trabalhos Acad\^emicos dispon\'ivel em \cite{UFJF2020}
\section{NOTAS} %%Nesta linha, dentro de { }, digita-se o nome da se\c{c}\~ao secund\'aria em CAIXA ALTA, como apresentado aqui.

Notas de rodap\'e s\~ao observa\c{c}\~{o}es e/ou aditamentos que o autor precisa incluir no texto \footnote[2]{As notas devem ser alinhadas sendo que na segunda linha da mesma nota, a primeira letra deve estar abaixo da primeira letra da primeira palavra da linha superior, destacando assim o expoente.}. Para a numera\c{c}\~{a}o das notas deve-se utilizar algarismos ar\'abicos. As notas devem ser digitadas dentro das margens, ficando separadas do texto por um espa\c{c}o simples entre as linhas e por filete de 5 cm a partir da margem esquerda e em fonte menor (um ponto) do corpo do texto. Observe os exemplos no Manual de Normaliza\c{c}\~{a}o de Trabalhos Acad\^emicos dispon\'ivel no \textit{link} abaixo: \\
\url{https://www2.ufjf.br/biblioteca/wp-content/uploads/sites/56/2020/08/Manual-2020-revisado.pdf}