\chapter{RESULTADOS E DISCUSS\~AO}
\label{cap:capitulo4}

\section{Desempenho dos modelos}
%Apresentar e comparar os resultados dos diferentes modelos utilizados.

Os resultados obtidos pelos modelos apresentaram variações significativas em termos de precisão, tanto nas previsões pontuais quanto nos intervalos de confiança, refletindo diferentes graus de precisão para cada cenário analisado. A análise dos resíduos revelou-se particularmente interessante, pois evidenciou comportamentos distintos em função da sazonalidade, com variações marcantes observadas em diferentes períodos do ano.

\subsection{Rio Jequitinhonha}

Iniciando pela menor bacia hidrográfica estudada, os resultados revelaram-se bastante satisfatórios, especialmente no que se refere ao modelo de Regressão Linear, que se destacou com previsões precisas, tanto em termos pontuais quanto nos intervalos de confiança. Este modelo demonstrou uma capacidade robusta de capturar a dinâmica hidrológica do rio, refletindo precisão nas métricas e desempenho consistente, aliado a um tempo de execução significativamente reduzido. Em contrapartida, o modelo (simples) SeasonalNaive apresentou resultados abaixo das expectativas em todas as situações avaliadas. (figura \ref{fig:jequiti_SN_WFV})

Com a MAPE extremamente elevada, de 150\%, os resultados indicaram um viés significativo de superestimação, conforme evidenciado pela métrica PBIAS. Verificando a KGE, ficou negativa. Em outras palavras, isso significa que o modelo não apenas falha em capturar a variabilidade dos dados observados, mas também introduz erros que o tornam menos eficaz do que uma abordagem simplista, como utiilizar a média histórica. Embora a análise superficial da qualidade dos intervalos de confiança pudesse sugerir um desempenho satisfatório do modelo, uma inspeção mais detalhada revela uma incongruência: os valores inferiores do intervalo (lo-95) foram calculados abaixo de zero, o que não faz sentido para o rio em questão, pois implicaria na ausência total de vazão, algo inviável para as condições medidas pela estação.

%Além disso, o atraso (\textit{delay}) da série prevista foi considerável, atingindo 56,88 dias, com um desvio-padrão de 61,3 dias, sugerindo que um evento pode demorar mais de 60 dias para ser refletido na previsão do modelo.

Considerando o desempenho insatisfatório do modelo SN, a análise foi encerrada neste ponto, sem proceder com a avaliação dos resíduos ou a análise da importância das variáveis. O modelo foi incluído apenas para fins comparativos. A análise mais aprofundada será dedicada aos modelos mais complexos, que apresentaram desempenho superior.

\begin{figure}[!h]
	\centering
	\includegraphics[scale=0.33]{Figuras/jequiti/wfv/SN/SN_WFV_ORIG.png}
	\caption{Resultado do SeasonalNaive no teste \textit{Walk-Forward Validation}\\(fonte: o autor)}
	\label{fig:jequiti_SN_WFV_ORIG}
\end{figure}

%\begin{table}[!h]
%	\centering \small
%	\caption{Resultados SeasonalNaive - rio Jequitinhonha \\(fonte: o autor)}
%	\begin{tabular}{|l|r|r|r|r|r|r|} \hline 
	%		\textbf{Horizonte} & \textbf{MAPE} & \textbf{RMSE} & \textbf{PBIAS} \\\hline
	%		1 dia              & 0,871         & 831,44        & -87,11 \\\hline
	%		3 dias             & 1,528         & 1872,25       & 102,71 \\\hline
	%		7 dias             & 1,046         & 1529,13       & 49,64  \\\hline
	%		15 dias            & 0,791         & 1468,02       & -13,46 \\\hline
	%	\end{tabular}
%	\label{tab:sn_jequitinhonha_resultados}
%\end{table}

Os resultados obtidos utilizando o modelo de Regressão Linear mostraram-se bastante promissores.(figura \ref{fig:jequiti_LR_WFV_LOG}) Nesta primeira avaliação, os dados foram log-transformados. Isso é verificável pelo título com o uso de ``destransformados''. Os dados foram log-transformados e retornados para a escala original para desenhar o gráfico e ficarem acessíveis para quem lê. \underline{Essa dinâmica no título se manterá por todo trabalho}.

Partindo pela KGE calculada, o resultado mostrou-se excelente. Recordando: quanto mais próxima de 1, melhor. Este valor sugere que o modelo é eficaz na previsão do comportamento hidrológico do sistema em análise, oferecendo previsões que estão bem alinhadas com os dados observados. A MAPE de $14,5\%$ sugere que o modelo tem uma precisão razoável e é bastante confiável. O modelo apresentou um viés sistemático de subestimar os resultados, conforme aponta a PBIAS de $-0,97\%$. Considerando todas as métricas, o resultado indica que o modelo teve um desempenho global muito bom. A KGE alta é particularmente indicativa de um bom ajuste global, com o modelo capturando bem tanto a dinâmica quanto a magnitude dos dados observados.

Considerando a qualidade dos intervalos de previsão, a cobertura observada de $98,08\%$ excedeu o intervalo teórico calculado de $95\%$, o que, à primeira vista, poderia ser interpretado como um desempenho satisfatório. No entanto, o limite superior do intervalo (hi-95) mostrou-se excessivamente elevado nos meses de janeiro e fevereiro, o que pode comprometer a interpretação dos resultados. Isso ocorre porque intervalos de previsão excessivamente amplos podem capturar praticamente qualquer valor observado, reduzindo a utilidade prática da previsão.

É importante destacar que os intervalos de previsão são calculados a partir dos erros do modelo durante a etapa de treinamento (\textit{in-sample residuals}). Para uma análise mais aprofundada desse comportamento, é necessário examinar os dados anteriores a 2023. A figura \ref{fig:jequiti_LR_final_2022_detalhe} revela que, nos últimos dias de 2022, houve observações significativamente elevadas. É provável que os erros associados a esse período mais próximo tenham influenciado a amplitude dos intervalos de previsão subsequentes. Essa interpretação é suportada pela observação de que o comportamento ao final de 2023 foi mais estável, possivelmente devido à ausência de eventos ruidosos imediatamente anteriores (meses de maio a agosto).

\begin{figure}[!h]
	\centering
	\includegraphics[scale=0.33]{Figuras/jequiti/wfv/LR/LR_WFV_LOG.png}
	\caption{\textit{Walk-Forward Validation} para o modelo Regressão Linear - LR\\(fonte: o autor)}
	\label{fig:jequiti_LR_WFV_LOG}
\end{figure}

\begin{figure}[!h]
	\centering
	\includegraphics[scale=0.33]{Figuras/jequiti/LR_final_2022_detalhe.png}
	\caption{Detalhe do trecho final dos dados, em 2022, usados para treinamento.\\(fonte: o autor)}
	\label{fig:jequiti_LR_final_2022_detalhe}
\end{figure}

%A análise de \textit{delay} mostrou um resultado médio de $-0,87$. Se pegar a série prevista pelo modelo e comprimir linearmente em $-0,87$, ambas as sequências serão idênticas, ou seja, a série prevista será a série observada. Considerando que os dados estão numa frequência diária, isso pode ser interpretado como o modelo atrasando a previsão em $0,87$ dias, o que seria menos de 24 horas entre o evento ocorrer e ele ser percebido na previsão. Afirmar que o modelo está atrasado 20,88 horas, talvez, não convirja com a realidade, por isso que se considerar o desvio-padrão no resultado, confere mais coerência, pois indicaria que o atraso percebido no fenômeno é de cerca de um dia e meio, dois dias.

\begin{figure}[!h]
	\centering
	\includegraphics[scale=0.33]{Figuras/jequiti/wfv/LR/LR_WFV_LOG_RESID_x_PREV.png}
	\caption{Dispersão dos resíduos.\\(fonte: o autor)}
	\label{fig:jequiti_LR_WFV_LOG_RESID_x_PREV}
\end{figure}

Na figura \ref{fig:jequiti_LR_WFV_LOG_RESID_x_PREV} podemos observar a concentração dos resíduos em torno de zero. A linha vermelha tracejada é onde o valor previsto e observado são iguais, designando a previsão perfeita, com resíduo $0$. Este é o comportamento que se espera, idealmente, os resíduos estarem distribuídos aleatoriamente em torno de zero, sem padrões evidentes (nenhuma tendência clara de curvatura ou cone). Contudo, à medida que se caminha sobre a linha vermelha tracejada, aumentando o valor no eixo x, há um aumento na dispersão dos resíduos. A área sombreada demarcada são os valores calculados para \textit{lower fence} e \textit{upper fence} a partir do primeiro ($-12,83$) e terceiro ($19,98$) quartil, que aqui resultaram em $-62,06$ e $69,21$, respectivamente. Entre estes valores é onde se espera que os resíduos estejam distribuídos (houve uma prevalência de $87,4\%$ dos resíduos nesta área), o que fica de fora dessas faixas pode ser interpretado como um \textit{outlier}. Considerando que os dados de treinamento não foram tratados para valores \textit{outliers}, isso parece estar se refletindo neste resultado, mesmo com a transformação logarítmica. O modelo não captou corretamente os valores elevados nos dados observados, por isso resíduos tão grandes assim, ainda que tenha apresentado um comportamento geral bom, como visto nas métricas anteriormente. Outro fator a se considerar também é que por estar na escala log, qualquer pequena variação nesta escala, quando retornada para a escala original, pode significar números elevados.

Veja na figura \ref{fig:jequiti_LR_WFV_LOG_RESID_x_TEMPO} como os resíduos estão dispersos ao longo do tempo. No início do ano e final do ano, exatamente quando os eventos de chuva mais ocorrem, o modelo tende a se dispersar. No início do ano, como discutido anteriormente, pode ser que devido às vazões elevadas do final do ano de 2022, isso tenha causado ruídos em excesso na previsão do modelo. Quando houve vazões mais moderadas, os resíduos foram também mais moderados, como pode-se observar no final do ano de 2023, em que a influência imediata é o meio do ano, meses de inverno, e início de primavera. Quando da estação de baixa dos rios, os meses de outono e inverno, os resíduos estiveram bem alinhados em torno do $0$. Para encerrar essa análise, a figura \ref{fig:jequiti_LR_WFV_LOG_RESID_x_CURVA_NORMAL} apresenta que os resíduos estão próximos da normalidade quanto à distribuição destes em torno de $0$, com uma assimetria de $0,48$. O valor positivo para assimetria sugere que houve casos em que o modelo subestimou os valores (existe uma cauda à direita do centro dos dados), o que combina com os gráficos anteriores de dispersão. Mas cabe destacar que os resíduos não estão significativamente desviados em nenhuma direção, apresentam distribuição aleatória e não sistemática e este é um comportamento desejado. A análise da função de autocorrelação (ACF) está na figura \ref{fig:jequiti_LR_WFV_LOG_RESID_ACF} e aqui é analisado se existe independência ou dependência temporal entre os resíduos. A área sombreada representa a faixa ideal de permanência dos resíduos, onde se espera que a maioria dos resíduos esteja concentrada caso não haja autocorrelação significativa. Algumas \textit{lags} estão fora destes limites (picos), mais precisamente, \textit{lags} que estão mais próximas da \textit{lag} de referência. Este comportamento indica que o modelo pode ser refinado para melhor capturar dinâmicas temporais, possivelmente incorporando termos de tendência. Mas no aspecto geral, está bom o comportamento dos resíduos. Para melhorar os resíduos com valores tão elevados um tratamento de \textit{outliers} pode verter bons resultados.

\begin{figure}[!h]
	\centering
	\includegraphics[scale=0.33]{Figuras/jequiti/wfv/LR/LR_WFV_LOG_RESID_x_TEMPO.png}
	\caption{Resíduos da previsão ao longo do tempo.\\(fonte: o autor)}
	\label{fig:jequiti_LR_WFV_LOG_RESID_x_TEMPO}
\end{figure}

\begin{figure}[!h]
	\centering
	\includegraphics[scale=0.33]{Figuras/jequiti/wfv/LR/LR_WFV_LOG_RESID_x_CURVA_NORMAL.png}
	\caption{Histograma dos resíduos.\\(fonte: o autor)}
	\label{fig:jequiti_LR_WFV_LOG_RESID_x_CURVA_NORMAL}
\end{figure}

\begin{figure}[!h]
	\centering
	\includegraphics[scale=0.33]{Figuras/jequiti/wfv/LR/LR_WFV_LOG_RESID_ACF.png}
	\caption{Resíduos da previsão ao longo do tempo.\\(fonte: o autor)}
	\label{fig:jequiti_LR_WFV_LOG_RESID_ACF}
\end{figure}
\clearpage

Agora os resultados sem emprego da log-transformação, com os dados originais. Salientando que para o modelo LR não é correto aplicar os dados originais na escala em que se encontram, foi precisar realizar normalização antes empregando algoritmo MinMax. Este algoritmo coloca todos os dados em valores entre $0$ e $1$, sendo o valor mínimo correspondente a cada série temporal passado para $0$ e o valor máximo é passado para $1$. Com os demais valores é feito, basicamente, uma regra de três para achar o valor correspondente na escala entre $0$ e $1$.

Dito isso, o resultado para os dados originais ficou levemente inferior ao modelo com os dados log-transformados. Em todas as métricas que se avaliar o resultado ficou piorado. Um comportamento destacado foi as faixas nos valores dos intervalos de previsão. Diferentemente do comportamento anterior (figura \ref{fig:jequiti_LR_WFV_LOG}), em que houve uma prevalência de faixas elevadas no início do ano, neste resultado o início do ano esteve, digamos, bem comportado. Ao decorrer do ano que os intervalos incrementaram, ali a partir do mês de fevereiro, e foram aumentando até o fim do ano. Isso pode ser um indicativo de que o modelo teve problemas na estabilidade de longo prazo, acumulando incertezas no período.

Ainda que tenha apresentado uma KGE inferior, bem como nas demais métricas, quando se observa o comportamento dos resíduos, houve uma prevalência de $91,78\%$ dos resíduos na área sombreada no gráfico de dispersão.(figura \ref{fig:jequiti_LR_WFV_SCLD_RESID_x_PREV}) Isso significa que houve menos \textit{outliers} nas previsões. Pode-se verificar também este resultado na figura \ref{fig:jequiti_LR_WFV_SCLD_RESID_x_TEMPO}, no início do ano, em que menos \textit{outliers} estão presentes, ainda que na porção final do ano tenha aparecido alguns a mais que não foram vistos na análise anterior (figura \ref{fig:jequiti_LR_WFV_LOG_RESID_x_TEMPO}). Para os dados originais, ainda que nas métricas pareça piorado, pela análise dos resíduos o modelo mostrou boa estabilidade nas previsões pontuais. Quando se considera os intervalos de previsão, precisa considerar com parcimônia, visto que o crescimento dos valores superioes (hi-95) e valores de vazão $0 m^3/s$ na faixa inferior (lo-95) indicam instabilidade de longo prazo.

%Para a análise de \textit{delay}, o mesmo feito anteriormente serve para este: aplicar o valor do desvio-padrão parece ser mais coerente com o comportamento real, sendo que aqui houve uma piora significativa no atraso, passando para mais de 3 dias ($3,78$).

Caminhando para o fim, o histograma e curva-normal apresentou uma assimetria consideravelmente superior ($3,37$) ao resultado de antes, com presença de cauda longa à direita.(figura \ref{fig:jequiti_LR_WFV_SCLD_RESID_x_CURVA_NORMAL}) O modelo teve tendência de subestimar os resultados, o que é verificável pelo PBIAS. Na figura \ref{fig:jequiti_LR_WFV_SCLD_RESID_ACF} é possível perceber picos fora do intervalo de confiança. Depois de cerca de 10 lags, os pontos de autocorrelação ficam dentro desta faixa, indicando que a maioria dos resíduos após esse ponto não está significativamente correlacionada com valores anteriores. Este pontos vermelhos no início do gráfico indicam que pode haver correlações significativas entre os resíduos com pequenos \textit{lags}, o que sugere, nos primeiros períodos, que os resíduos estão correlacionados com os valores anteriores. Isso é indicativo de que o modelo pode não estar capturando totalmente a estrutura temporal dos dados. Pode ser necessário ajustar o modelo, adicionar variáveis que expliquem essa dependência temporal, ainda que variáveis de valor acumulado tenham sido inseridas exatamente na intenção de capturar tais comportamentos. Mas claramente precisaria aprimorar.

\begin{figure}[!h]
\centering
\includegraphics[scale=0.33]{Figuras/jequiti/wfv/LR/LR_WFV_ORIG.png}
\caption{\textit{Walk-Forward Validation} para o modelo Regressão Linear - LR.\\(fonte: o autor)}
\label{fig:jequiti_LR_WFV_ORIG}
\end{figure}

\begin{figure}[!h]
\centering
\includegraphics[scale=0.33]{Figuras/jequiti/wfv/LR/LR_WFV_ORIG_RESID_x_PREV.png}
\caption{Dispersão dos resíduos.\\(fonte: o autor)}
\label{fig:jequiti_LR_WFV_ORIG_RESID_x_PREV}
\end{figure}

\begin{figure}[!h]
\centering
\includegraphics[scale=0.33]{Figuras/jequiti/wfv/LR/LR_WFV_ORIG_RESID_x_TEMPO.png}
\caption{Dispersão dos resíduos ao longo do ano.\\(fonte: o autor)}
\label{fig:jequiti_LR_WFV_ORIG_RESID_x_TEMPO}
\end{figure}

\begin{figure}[!h]
\centering
\includegraphics[scale=0.33]{Figuras/jequiti/wfv/LR/LR_WFV_ORIG_RESID_x_CURVA_NORMAL.png}
\caption{Histograma e curva-normal dos resíduos.\\(fonte: o autor)}
\label{fig:jequiti_LR_WFV_ORIG_RESID_x_CURVA_NORMAL}
\end{figure}

\begin{figure}[!h]
\centering
\includegraphics[scale=0.33]{Figuras/jequiti/wfv/LR/LR_WFV_ORIG_RESID_ACF.png}
\caption{Gráfico ACF dos resíduos.\\(fonte: o autor)}
\label{fig:jequiti_LR_WFV_ORIG_RESID_ACF}
\end{figure}
\clearpage

Passando agora à análise dos modelos principais deste trabalho, cujos resultados serão comparados ao modelo de referência, optou-se por realizar a análise de forma concomitante, uma vez que ambos os modelos apresentaram comportamentos similares, embora o modelo RandomForest tenha demonstrado um desempenho superior ao CatBoost.

Na métrica KGE, comparando-se os resultados com o comportamento ilustrado na figura \ref{fig:jequiti_LR_WFV_LOG}, observa-se que ambos os modelos não conseguiram superar o modelo de referência, com destaque para o CB, que apresentou uma performance consideravelmente inferior. No entanto, ao considerar a precisão percentual média, avaliada pela métrica MAPE, ambos os modelos baseados em árvores apresentaram melhorias em relação ao modelo de referência, evidenciando um desempenho superior em termos de erro percentual médio.

A KGE, relembrando, combina três aspectos fundamentais: variabilidade, viés e correlação entre os dados observados e previstos. Considerando esses fatores, com os dados log-transformados, o modelo LR capturou de forma mais eficaz os três aspectos mencionados. É provável que a linearização tenha sido um fator determinante para o bom desempenho deste modelo olhando por esta métrica, uma vez que, ao analisar a MAPE, os modelos não-lineares (CB e RF) apresentaram desempenhos melhores. Um desempenho médio percentual melhor pode dever-se à resiliência dos modelos não-lineares à sua robustez diante de valores discrepantes, aos quais o modelo linear é mais sensível.

Observa-se que tanto o CB quanto o RF apresentaram intervalos de previsão menos amplos no início do ano, em comparação com o modelo LR, mesmo impactados pelas vazões elevadas no final de 2022 (figura \ref{fig:jequiti_LR_final_2022_detalhe}). Em termos de previsão pontual, ambos os modelos não lineares demonstraram-se robustos. No entanto, ao analisar a cobertura empírica dos intervalos de previsão, o modelo RF mostrou-se consideravelmente defasado em relação aos modelos LR e CB. Isso pode ser explicado pela possível ``otimização'' excessiva do modelo ao calcular os intervalos, ao presumir uma repetição dos eventos e erros passados, resultando em intervalos estreitos. Esse fenômeno é descrito por \citet{RobHyndman_prediction_intervals}. Apesar disso, a cobertura empírica de $80\%$ do modelo RF ainda pode ser considerada satisfatória, e a cobertura de $88\%$ obtida pelo CB demonstra um bom desempenho.

Pode-se inferir que os modelos CB e RF conseguiram equilibrar a previsão pontual, indicada pela MAPE de $0,12$, com os intervalos de previsão. Um ajuste de hiperparâmetros poderia potencialmente melhorar esse desempenho. Em relação ao PBIAS, ambos os modelos apresentaram um desvio sistemático, subestimando os valores previstos.

%No que diz respeito ao \textit{delay}, todos os modelos, incluindo o LR, apresentaram um comportamento semelhante, com um atraso de aproximadamente 1 a 2 dias para que um evento na série observada fosse captado nas previsões. Essa análise leva em consideração o desvio-padrão.

\begin{figure}[!h]
\centering
\includegraphics[scale=0.33]{Figuras/jequiti/wfv/CB/CB_WFV_LOG.png}
\caption{\textit{Walk-Forward Validation} para o modelo CatBoost - CB.\\(fonte: o autor)}
\label{fig:jequiti_CB_WFV_LOG}
\end{figure}

\begin{figure}[!h]
\centering
\includegraphics[scale=0.33]{Figuras/jequiti/wfv/RF/RF_WFV_LOG.png}
\caption{\textit{Walk-Forward Validation} para o modelo RandomForest - RF.\\(fonte: o autor)}
\label{fig:jequiti_RF_WFV_LOG}
\end{figure}
\clearpage

Os resíduos em modelos não-lineares podem ser mais difíceis de interpretar porque esses modelos capturam interações e padrões complexos. Porém, mesmo assim, detectar padrões sistemáticos nos resíduos ajuda a elucidar se os modelos conseguiram capturar todas as nuances dos dados.

Em ambos os casos, não houve prevalência de comportamento anormal para os resíduos. Estiveram aleatoriamente distribuídos em torno de $0$, com uma menção importante para a maior dispersão de possíveis valores \textit{outliers} para o modelo CB à medida que as medições aumentaram.(figura \ref{fig:jequiti_CB_WFV_LOG_RESID_x_PREV}).

\begin{figure}[!h]
\centering
\includegraphics[scale=0.33]{Figuras/jequiti/wfv/CB/CB_WFV_LOG_RESID_x_PREV.png}
\caption{Dispersão dos resíduos.\\(fonte: o autor)}
\label{fig:jequiti_CB_WFV_LOG_RESID_x_PREV}
\end{figure}

\begin{figure}[!h]
\centering
\includegraphics[scale=0.33]{Figuras/jequiti/wfv/RF/RF_WFV_LOG_RESID_x_PREV.png}
\caption{Dispersão dos resíduos.\\(fonte: o autor)}
\label{fig:jequiti_RF_WFV_LOG_RESID_x_PREV}
\end{figure}
\clearpage

A dispersão ao longo do tempo foi praticamente a mesma para ambos os modelos, com resíduos de valor elevado mais presentes no início da série. (figuras \ref{fig:jequiti_CB_WFV_LOG_RESID_x_TEMPO} \ref{fig:jequiti_RF_WFV_LOG_RESID_x_TEMPO}) Aqui vale a interpretação usada para o modelo de referência, de que os valores elevados aferidos em final de 2022 possam ter interferido. Houve uma prevalência de $84,11\%$ e $86,03\%$, respectivamente CB e RF, dos resíduos na área sombreada, correspondente à área de erro aceitável. Uma leve melhor prevalência para o modelo RF, que pode ser vista também em menos dispersão de resíduos para fora dos limites da área sombreada no início do ano.

\begin{figure}[!h]
\centering
\includegraphics[scale=0.33]{Figuras/jequiti/wfv/CB/CB_WFV_LOG_RESID_x_TEMPO.png}
\caption{Dispersão dos resíduos ao longo do ano.\\(fonte: o autor)}
\label{fig:jequiti_CB_WFV_LOG_RESID_x_TEMPO}
\end{figure}

\begin{figure}[!h]
\centering
\includegraphics[scale=0.33]{Figuras/jequiti/wfv/RF/RF_WFV_LOG_RESID_x_TEMPO.png}
\caption{Dispersão dos resíduos ao longo do ano.\\(fonte: o autor)}
\label{fig:jequiti_RF_WFV_LOG_RESID_x_TEMPO}
\end{figure}
\clearpage

Em termos de assimetria da distribuição dos resíduos, ambos modelos comportaram-se iguais, com assimetria positiva indicando cauda à direita. Com valores de $2,56$ para o modelo CB e $2,54$ para o RF, isso mostra tendência de subestimar as previsões (visto no PBIAS de ambos os modelos).

\begin{figure}[!h]
\centering
\includegraphics[scale=0.33]{Figuras/jequiti/wfv/CB/CB_WFV_LOG_RESID_x_CURVA_NORMAL.png}
\caption{Histograma e curva-normal dos resíduos.\\(fonte: o autor)}
\label{fig:jequiti_CB_WFV_LOG_RESID_x_CURVA_NORMAL}
\end{figure}

\begin{figure}[!h]
\centering
\includegraphics[scale=0.33]{Figuras/jequiti/wfv/RF/RF_WFV_LOG_RESID_x_CURVA_NORMAL.png}
\caption{Histograma e curva-normal dos resíduos.\\(fonte: o autor)}
\label{fig:jequiti_RF_WFV_LOG_RESID_x_CURVA_NORMAL}
\end{figure}
\clearpage

Observando os gráficos de autocorrelação dos modelos, houve um comportamento parecido nas \textit{lags} iniciais. Nas primeiras \textit{lags}, os gráficos mostram pontos de autocorrelação que ultrapassam a faixa de confiança (sombreada em azul), indicando que os resíduos possuem dependência temporal significativa nos primeiros períodos. Esse padrão sugere que os modelos não conseguiram capturar totalmente a estrutura temporal dos dados, resultando em resíduos correlacionados em curtos períodos de tempo. Uma possível explicação pode ser a presença de componentes sazonais ou padrões de curto prazo que não foram completamente modelados.

A partir de aproximadamente a \textit{lag} 20, a autocorrelação dos resíduos cai para valores próximos de zero e permanece dentro da faixa de confiança para as \textit{lags} seguintes até a \textit{lag} 365. Este é um comportamento esperado para um bom modelo, onde os resíduos devem ser independentes e distribuídos aleatoriamente ao longo do tempo. A presença de resíduos com baixa autocorrelação em \textit{lags} maiores indica que, em períodos mais longos, os modelos não apresentam problemas de dependência temporal.

\begin{figure}[!h]
\centering
\includegraphics[scale=0.33]{Figuras/jequiti/wfv/CB/CB_WFV_LOG_RESID_ACF.png}
\caption{Gráfico ACF dos resíduos.\\(fonte: o autor)}
\label{fig:jequiti_CB_WFV_LOG_RESID_ACF}
\end{figure}

\begin{figure}[!h]
\centering
\includegraphics[scale=0.33]{Figuras/jequiti/wfv/RF/RF_WFV_LOG_RESID_ACF.png}
\caption{Gráfico ACF dos resíduos.\\(fonte: o autor)}
\label{fig:jequiti_RF_WFV_LOG_RESID_ACF}
\end{figure}
\clearpage

Neste momento será apresentado o resultado dos modelos CB e RF para os dados em sua escala original, sem a log-transformação. Recapitulando: a log-transformação lineariza os dados e para o modelo LR ela realmente resultou em melhora do comportamento do modelo. Esperava-se que o mesmo pudesse, eventualmente, ocorrer com estes modelos não-lineares, pois simplificaria as relações entre as variáveis. Porém, aconteceu o oposto. Permanecer com os dados originais, \textit{in natura} por assim dizer, foi o que fez ambos se comportarem melhor.

Veja o comportamento do \textit{walk-forward validation}. Houve melhoras em todas métricas nos dois modelos, com exceção para o modelo RF, que a MAPE piorou com os dados originais em relação aos dados log-transformados, ficando $1,4\%$ acima. É ínfimo, porém não descartável. Mas a KGE, uma métrica mais robusta, melhorou bastante entre o resultado anterior e este resultado em tela. Aplicando os modelos CB e RF nos dados originais, melhorou também o viés sistemático (PBIAS). O modelo CB continuou mostrando tendência de subestimar as previsões mas aqui o modelo RF inverteu, passando a apresentar tendência de superestimar as previsões. A cobertura empírica dos intervalos de previsão melhoraram, sendo o CB pouco mais de $1\%$ melhor, mas o RF melhorou mais de $4\%$ nos cálculos dos intervalos. À respeito dos intervalos de previsão, os modelos mostraram tendência de aumento das faixas ao longo do ano. Diferentemente da análise log-transformada, aqui os modelos parecem ter demostrado incerteza quanto aos erros calculados em eventos anteriores, ainda que no início do ano, para ambos, pareça ter ficado mais estável o cálculo dos intervalos.

\begin{figure}[!h]
\centering
\includegraphics[scale=0.33]{Figuras/jequiti/wfv/CB/CB_WFV_ORIG.png}
\caption{\textit{Walk-Forward Validation} para o modelo CatBoost - CB.\\(fonte: o autor)}
\label{fig:jequiti_CB_WFV_ORIG}
\end{figure}

\begin{figure}[!h]
\centering
\includegraphics[scale=0.33]{Figuras/jequiti/wfv/RF/RF_WFV_ORIG.png}
\caption{\textit{Walk-Forward Validation} para o modelo RandomForest - RF.\\(fonte: o autor)}
\label{fig:jequiti_RF_WFV_ORIG}
\end{figure}
\clearpage

Observa-se que a maior parte dos resíduos está concentrada em torno da linha de referência $y=0$, o que indica que os modelos CatBoost foram capazes de produzir previsões relativamente boas para a maioria dos dados. Ambos apresentaram também valores extremos, \textit{outliers}, e essa presença pode indicar que os modelos enfrentaram dificuldades em prever corretamente alguns valores mais extremos, resultando em erros maiores para certas previsões. O modelo RF apresenta uma distribuição de resíduos um pouco mais ampla que o CB, com alguns pontos chegando a resíduos superiores a $400$ e abaixo de $-300$.

Ambos os modelos apresentaram bom desempenho geral, com a maioria dos resíduos concentrados em torno da linha de $y=0$, indicando previsões razoáveis. No entanto, também apresentaram \textit{outliers} e tendência dos resíduos aumentarem em magnitude para previsões maiores. Isso indica que tanto o modelo CB quanto o RF podem enfrentar dificuldades em prever valores extremos com precisão, um comportamento já demonstrado anteriormente.

\begin{figure}[!h]
\centering
\includegraphics[scale=0.33]{Figuras/jequiti/wfv/CB/CB_WFV_ORIG_RESID_x_PREV.png}
\caption{Dispersão dos resíduos.\\(fonte: o autor)}
\label{fig:jequiti_CB_WFV_ORIG_RESID_x_PREV}
\end{figure}

\begin{figure}[!h]
\centering
\includegraphics[scale=0.33]{Figuras/jequiti/wfv/RF/RF_WFV_ORIG_RESID_x_PREV.png}
\caption{Dispersão dos resíduos.\\(fonte: o autor)}
\label{fig:jequiti_RF_WFV_ORIG_RESID_x_PREV}
\end{figure}
\clearpage

A maior parte dos resíduos está concentrada em torno da linha de referência $y=0$, indicando que os modelos mantiveram um desempenho geral estável ao longo do tempo. Porém, há \textit{outliers} significativos no início de 2023 (valores de resíduos acima de $400$ e abaixo de $-400$ para o CB e acima de $500$ e abaixo de $-300$ para o RF). Esses outliers indicam que, durante o início do ano, ambos tiveram dificuldades em prever alguns eventos específicos, resultando em erros grandes.

Para o modelo CB, após fevereiro de 2023 (março para o modelo RF), os resíduos parecem estar melhor distribuídos dentro da faixa delimitada pelas faixas, com menos \textit{outliers} extremos, sugerindo que os modelos ajustaram-se melhor aos dados ao longo do tempo.

Em geral, ambos os modelos apresentaram um padrão de melhora ao longo do ano, com resíduos mais bem distribuídos após o primeiro trimestre de 2023. No entanto, os \textit{outliers} iniciais indicam que os modelos podem ter dificuldades com eventos específicos de sazonalidade ou anomalias no início do ano. O CB parece ter uma leve vantagem em termos de previsões mais consistentes ao longo do tempo (valores menos extremos). Ajustes de hiperparâmetros podem ajudar os modelos nos períodos mais desafiadores do ano, como no início do ano, ou adição de componentes de tendência e sazonalidade localizada.

\begin{figure}[!h]
\centering
\includegraphics[scale=0.33]{Figuras/jequiti/wfv/CB/CB_WFV_ORIG_RESID_x_TEMPO.png}
\caption{Dispersão dos resíduos ao longo do ano.\\(fonte: o autor)}
\label{fig:jequiti_CB_WFV_ORIG_RESID_x_TEMPO}
\end{figure}

\begin{figure}[!h]
\centering
\includegraphics[scale=0.33]{Figuras/jequiti/wfv/RF/RF_WFV_ORIG_RESID_x_TEMPO.png}
\caption{Dispersão dos resíduos ao longo do ano.\\(fonte: o autor)}
\label{fig:jequiti_RF_WFV_ORIG_RESID_x_TEMPO}
\end{figure}
\clearpage

A distribuição dos resíduos do modelo CB (figura \ref{fig:jequiti_CB_WFV_ORIG_RESID_x_CURVA_NORMAL}) apresenta uma assimetria positiva de $1,33$, o que indica que a cauda direita da distribuição é mais longa que a esquerda. Isso significa que há um maior número de resíduos positivos mais elevados (erros positivos grandes) do que negativos. A curva normal ajustada (em vermelho) não se ajusta perfeitamente à distribuição observada, especialmente nas caudas, o que indica que a distribuição dos resíduos não é perfeitamente normal. Esse comportamento é esperado, dado o valor da assimetria. A maior parte dos resíduos está concentrada em torno de $y=0$, com uma contagem elevada próxima a zero. Isso é um bom sinal, pois sugere que o modelo fez previsões razoavelmente precisas para a maior parte dos dados.
No entanto, a presença de resíduos elevados tanto na cauda direita (acima de $200$) quanto na cauda esquerda (abaixo de $-100$) indica que o modelo teve dificuldades em prever com precisão alguns valores mais extremos. A distribuição assimétrica com cauda longa à direita indica que o modelo CB teve mais dificuldades com previsões que resultaram em grandes erros positivos. Esses erros podem ser atribuídos a eventos atípicos ou extremos nos dados, que o modelo não conseguiu prever corretamente.

A distribuição dos resíduos do modelo RF (\ref{fig:jequiti_RF_WFV_ORIG_RESID_x_CURVA_NORMAL}) apresenta uma assimetria positiva de $0,49$, o que indica que a cauda direita é um pouco mais longa, mas a assimetria é menos pronunciada do que no modelo CB. A curva normal ajustada se aproxima mais da distribuição observada do que no gráfico do CB, o que sugere que os resíduos do RF seguem uma distribuição mais próxima de uma normal, apesar da pequena assimetria positiva. Assim como no gráfico do CB, a maior parte dos resíduos está concentrada em torno de zero, com a maioria das previsões sendo razoavelmente precisas. No entanto, o RF apresenta uma distribuição ligeiramente mais equilibrada, com menos resíduos extremos do que o CB. As caudas, tanto à esquerda quanto à direita, são menos pronunciadas, com menos resíduos extremos (acima de $200$ e abaixo de $-200$). Isso indica que o RF teve um desempenho mais estável e com menos erros em eventos extremos.

O RF tem um comportamento de resíduos mais próximo de uma distribuição normal, com menos assimetria e menos resíduos extremos. Isso indica que o modelo é mais estável e robusto em termos de previsões para a maioria dos dados. O CB mostrou maior sensibilidade a eventos extremos, com resíduos mais dispersos e uma assimetria maior, indicando que o modelo enfrenta dificuldades em capturar corretamente os valores mais extremos, resultando em erros maiores. Mas em geral, ambos os modelos apresentam uma boa concentração de resíduos em torno de zero, mostrando que a maioria das previsões está correta, ainda que o CB tenha se apresentado mais suscetível a \textit{outliers}.

\begin{figure}[!h]
\centering
\includegraphics[scale=0.33]{Figuras/jequiti/wfv/CB/CB_WFV_ORIG_RESID_x_CURVA_NORMAL.png}
\caption{Histograma e curva-normal dos resíduos.\\(fonte: o autor)}
\label{fig:jequiti_CB_WFV_ORIG_RESID_x_CURVA_NORMAL}
\end{figure}

\begin{figure}[!h]
\centering
\includegraphics[scale=0.33]{Figuras/jequiti/wfv/RF/RF_WFV_ORIG_RESID_x_CURVA_NORMAL.png}
\caption{Histograma e curva-normal dos resíduos.\\(fonte: o autor)}
\label{fig:jequiti_RF_WFV_ORIG_RESID_x_CURVA_NORMAL}
\end{figure}
\clearpage

Nas primeiras \textit{lags}, há autocorrelações significativas que ultrapassam a faixa de confiança, especialmente nas primeiras 10 \textit{lags}. Isso indica que, para curtos intervalos de tempo, os resíduos do modelo CB estão correlacionados, sugerindo que o modelo deixou de capturar padrões de curto prazo.(figura \ref{fig:jequiti_CB_WFV_ORIG_RESID_ACF}) A presença dessas correlações positivas nos primeiros lags pode sugerir que há padrões temporais nos dados que o modelo CB não capturou adequadamente. No entanto, a partir de aproximadamente a \textit{lag} 20, as autocorrelações caem para valores dentro da faixa de confiança de $95\%$ e permanecem próximas de zero. Isso sugere que, para intervalos maiores, os resíduos não apresentam correlação significativa, o que é um sinal positivo de que, em termos de longo prazo, o modelo está capturando a variabilidade dos dados de forma satisfatória.

Assim como no gráfico do CB, o modelo RF também apresenta autocorrelações significativas nas primeiras \textit{lags} (até cerca de 10 \textit{lags}).(figura \ref{fig:jequiti_RF_WFV_ORIG_RESID_ACF}) Isso indica uma dependência temporal de curto prazo que o modelo RF não capturou completamente, resultando em resíduos correlacionados em intervalos curtos. No entanto, as correlações significativas parecem ser um pouco mais dispersas em comparação com o modelo CB, o que pode indicar um comportamento ligeiramente melhor para curtos intervalos no modelo RF. A partir de aproximadamente a \textit{lag} 20, as autocorrelações do RF caem para dentro da faixa de confiança e permanecem próximas de zero, assim como no modelo CB, indicando que, para previsões de longo prazo, o modelo RF está capturando bem a estrutura dos dados e os resíduos se comportam de maneira aleatória.

Ambos os modelos CB e RF apresentam autocorrelações significativas nas primeiras \textit{lags}, denotando que nenhum modelo conseguiu capturar completamente a dependência temporal de curto prazo nos dados. No entanto, o RF teve uma dispersão um pouco menor de correlações significativas em comparação com o CB, o que indica um desempenho ligeiramente melhor no curto prazo.

Para \textit{lags} maiores (acima de 20), tanto o CB quanto o RF apresentaram resíduos que se comportaram de maneira aleatória, com autocorrelações dentro da faixa de confiança e próximas de zero. Considernado as previsões de longo prazo, ambos os modelos estão funcionando de forma satisfatória e não apresentaram padrões remanescentes nos resíduos.

\begin{figure}[!h]
\centering
\includegraphics[scale=0.33]{Figuras/jequiti/wfv/CB/CB_WFV_ORIG_RESID_ACF.png}
\caption{Gráfico ACF dos resíduos.\\(fonte: o autor)}
\label{fig:jequiti_CB_WFV_ORIG_RESID_ACF}
\end{figure}

\begin{figure}[!h]
\centering
\includegraphics[scale=0.33]{Figuras/jequiti/wfv/RF/RF_WFV_ORIG_RESID_ACF.png}
\caption{Gráfico ACF dos resíduos.\\(fonte: o autor)}
\label{fig:jequiti_RF_WFV_ORIG_RESID_ACF}
\end{figure}
\clearpage

Para o rio Jequitinhonha a análise termina aqui. Aplicar a log-transformação melhorou o resultado do modelo LR de quando não aplicada essa transformação, e os modelos CB e RF (não-lineares) foram melhores quando os dados não foram transformados. Aqui, para este rio, foi este o resultado, mais adiante será visto para os demais rios.

Uma coisa que precisa ficar claro aqui: só serão discutidos para os outros rios os melhores resultados. Para o Jequitinhonha foi mostrada toda linha de trabalho para que ficasse claro que toda comparação foi realizada entre resultados log-transformados e não transformados, tudo foi estudado e analisado. Mas pelo bem da concisão (e deve ter ficado evidente as voltas dadas no texto mostrando exatamente cada resultado), todo resultado não mostrado neste capítulo será colocado em um apêndice para que o leitor e a leitora possam averiguar a corretude do trabalho. E não será mais incluída a análise para o modelo SeasonalNaive.

Adiante.

\subsection{Rio Doce}

Os resultados para o rio Doce apresentaram-se muito bons, considerando-se todas as métricas do trabalho. Isso possivelmente se deve à qualidade dos dados de treinamento, em que não foi preciso preencher tantos dados faltantes de vazão. Relembre: de $4017$ registros, apenas $3$ faltavam.

O modelo CB apresentou um desempenho muito bom, com a KGE de $0,89$ sugerindo alta correlação entre os valores previstos e os observados. Quando se considera a MAPE, foi a mais baixa entre os três modelos, contudo, mesmo assim, indica uma precisão elevada nas previsões médias, o que demonstra robustez do modelo. A PBIAS de $-3.73\%$ indica um leve viés geral para subestimar as previsões, e a cobertura empírica de $91,23\%$ mostra que o intervalo de previsão captura bem a variabilidade dos dados observados.

O gráfico \ref{fig:doce_CB_WFV_ORIG} revela ainda que o modelo acompanha bem os picos de vazão e que os intervalos de previsão foram particularmente estreitos quando no período de vazões elevadas e relativamente amplos nos períodos de menores vazões. Indica que as medições de precipitação nos períodos chuvosos conferem mais precisão ao comportamento médio do modelo. Em outras palavras: o modelo fica ``otimista'' quando os dados de chuva não estão zerados e possui mais informações, o que ocorre nos períodos de outono e inverno. Tanto que nestes períodos mencionados, quando há uma profusão de valores $0$ na precipitação, os intervalos de previsão ficaram amplos, o que denota a tentativa do modelo de ``acertar'' a medida. Comportamento este que o modelo RF também apresentou em relação aos intervalos de previsão - visto em \ref{fig:doce_RF_WFV_ORIG}. A cobertura empírica dos intervalos de previsão de ambos os modelos ficaram próximas, com uma leve vantagem para o modelo CB - o modelo RF obteve $88,77\%$.

O modelo LR, mesmo sendo um modelo simples, apresentou um desempenho notável, com uma MAPE ligeiramente maior, de $0,084$ do que a do CB, mas ainda assim baixo. A KGE foi a maior dos três modelos, alcançando $0,94$, mostrando uma ótima correlação e menor erro de variância entre os valores observados e os previstos. A PBIAS de $-2,75\%$ sugere um viés de subestimação, sendo o ``caminho do meio'' entre os modelos. A cobertura empírica de $96,71\%$ indica que o intervalo de previsão captura quase a totalidade da variabilidade dos dados observados - visível no gráfico \ref{fig:doce_LR_WFV_LOG}. O gráfico também mostra que o LR também consegue capturar os picos de vazão de forma eficaz, com intervalos de previsão mais estreitos ao longo do ano se comparados ao CB e ao RF.

O modelo RF também teve um bom desempenho, com uma MAPE intermediária entre os outros dois modelos e uma KGE de $0,93$, ficando muito próximo ao do LR. A PBIAS de $-1,44\%$ indica o menor viés de subestimação entre os três modelos, no entanto, a cobertura empírica foi a menor, com $88,77\%$, sugerindo que o intervalo de previsão não abrange tanto a variabilidade observada quanto nos outros modelos. Isso é perceptível principalmente no início da série, nos meses de janeiro, fevereiro e março. Ainda que tenha captado bem os picos de vazão, um comportamento semelhante ao CB em termos de previsões, os intervalos de previsão ficaram aquém.

Em termos de precisão média (MAPE), o CB apresentou o melhor desempenho - valor de $0,082$ -, seguido de perto por RF - valor de $0,083$ - e LR - MAPE de $0,084$. A diferença foi muito pequena, na casa de $10^{-3}$, mostrando que todos os modelos tiveram alta precisão. Analisando a KGE - a mais complexa das métricas, que avalia a correlação, a variabilidade e o viés -, o modelo LR foi o melhor nesse critério, seguido pelo RF e CB. Isso sugere que, apesar de ser um modelo linear, a regressão conseguiu capturar bem a dinâmica não-linear das vazões do rio. Mas, novamente, os valores ficaram muito próximos, não havendo discrepância entre as medições.

\begin{itemize}
\item No geral, todos os três modelos apresentam bons resultados, mas com características específicas:
\begin{itemize}
\item O modelo CB saiu-se ligeiramente melhor em precisão - MAPE -, mas com um viés de subestimação mais acentuado;
\item O modelo LR surpreendentemente teve o melhor desempenho na KGE, um viés mediano e a melhor cobertura empírica; e
\item O modelo RF mostrou-se competitivo em precisão e viés, porém, com uma menor cobertura empírica, o que sugere que os intervalos de previsão são menos eficazes para capturar a incerteza nas previsões quando comparado aos demais modelos.
\end{itemize}
\end{itemize}

\begin{figure}[!h]
\centering
\includegraphics[scale=0.33]{Figuras/rio_doce/wfv/CB/CB_WFV_ORIG.png}
\caption{\textit{Walk-Forward Validation} para o modelo CatBoost - CB.\\(fonte: o autor)}
\label{fig:doce_CB_WFV_ORIG}
\end{figure}

\begin{figure}[!h]
\centering
\includegraphics[scale=0.33]{Figuras/rio_doce/wfv/LR/LR_WFV_LOG.png}
\caption{\textit{Walk-Forward Validation} para o modelo LinearRegression - LR.\\(fonte: o autor)}
\label{fig:doce_LR_WFV_LOG}
\end{figure}

\begin{figure}[!h]
\centering
\includegraphics[scale=0.33]{Figuras/rio_doce/wfv/RF/RF_WFV_ORIG.png}
\caption{\textit{Walk-Forward Validation} para o modelo RandomForest - RF.\\(fonte: o autor)}
\label{fig:doce_RF_WFV_ORIG}
\end{figure}
\clearpage

O gráfico \ref{fig:doce_CB_WFV_ORIG_RESID_x_PREV} do modelo CB mostra uma concentração de resíduos em torno de $0$ para previsões de até aproximadamente $500 m^3/s$, o que indica que o modelo está prevendo bem para valores mais baixos de vazão. No entanto, à medida que os valores previstos aumentam, os resíduos começam a se espalhar mais, o que indica que o modelo tem uma maior dificuldade em capturar picos de vazão mais altos.

Existem alguns resíduos que excedem os limites de $1,5 IQR$ (\textit{Interquartile Range}), especialmente para previsões mais altas, com resíduos acima de $100$ e abaixo de $-200$, sugerindo a presença de outliers em picos de vazão. E não parece haver uma tendência clara nos resíduos, o que é positivo, pois indica que o modelo não está sistematicamente superestimando ou subestimando em uma faixa específica de valores previstos.

Para o LR, observa-se no gráfico \ref{fig:doce_LR_WFV_LOG_RESID_x_PREV} um comportamento semelhante ao do CB no que se refere à concentração de resíduos em torno de $0$ para valores previstos mais baixos. Contudo, a dispersão dos resíduos é ligeiramente maior para previsões acima de $1000 m^3/s$, indicando uma leve perda de precisão em valores mais altos.

O modelo LR também mostra alguns \textit{outliers} em valores previstos mais altos, especialmente entre $2500$ e $4000 m^3/s$. Isso é esperado, dado que a regressão linear pode ter dificuldades em capturar não-linearidades e picos extremos de vazão. Repete-se aqui o mesmo comportamento do CB e não há uma tendência evidente, o que significa que o modelo está relativamente bem ajustado, mas com uma dispersão maior à medida que os valores previstos aumentam.

O RF apresentou um padrão semelhante aos dos modelos anteriores. A concentração dos resíduos em torno de $0$ para previsões menores também está presente, mas há uma maior dispersão dos resíduos para previsões acima de $1000 m^3/s$. Isso pode indicar que, assim como os outros modelos, o RF apresentou dificuldade em capturar corretamente alguns picos mais altos de vazão. Vale salientar que todos os modelos apresentaram viés geral sistemático de subestimação das previsões

Notam-se \textit{outliers} para valores previstos mais altos, com resíduos que variam de $300$ a mais de $1000$ para previsões maiores que $2500 m^3/s$. Esses \textit{outliers} são mais pronunciados que nos outros dois modelos. Estes valores elevados indicando outliers denotam que este modelo, bem como os outros anteriores, perde precisão em alguns valores de pico de vazão observados. Por fim, não houve uma tendência clara nos resíduos.

Em geral, todos os três modelos apresentam uma concentração de resíduos razoavelmente próxima de $0$ para valores de vazão previstos abaixo de $1000 m^3/s$ - prevalência de $87,4\%$ para resíduos do CB e RF e de $88,22\%$ para LR dentro da faixa $1,5 \pm IQR$ -, sugerindo que eles conseguem modelar bem as vazões mais comuns e de baixa intensidade. À medida que os valores previstos aumentam, todos os modelos mostram uma dispersão crescente nos resíduos, com picos de vazão sendo mais difíceis de prever corretamente. Contudo, o RF parece ter a maior dispersão, seguido por LR, enquanto que o CB apresentou uma dispersão ligeiramente menor, embora ainda presente.

Os três modelos possuem \textit{outliers}, especialmente em previsões mais altas, mostrando que isso é um desafio em dados de vazão de rios, que apresentam picos esporádicos. O RF tem os \textit{outliers} mais extremos, seguidos por LR e, por fim, CB. A crescente dispersão dos resíduos em valores previstos mais altos sugere que \underline{todos os modelos} enfrentaram desafios em prever valores de vazão muito elevados, ainda que tenham acompanhado os picos de vazão.

\begin{figure}[!h]
\centering
\includegraphics[scale=0.33]{Figuras/rio_doce/wfv/CB/CB_WFV_ORIG_RESID_x_PREV.png}
\caption{Dispersão dos resíduos.\\(fonte: o autor)}
\label{fig:doce_CB_WFV_ORIG_RESID_x_PREV}
\end{figure}

\begin{figure}[!h]
\centering
\includegraphics[scale=0.33]{Figuras/rio_doce/wfv/LR/LR_WFV_LOG_RESID_x_PREV.png}
\caption{Dispersão dos resíduos.\\(fonte: o autor)}
\label{fig:doce_LR_WFV_LOG_RESID_x_PREV}
\end{figure}

\begin{figure}[!h]
\centering
\includegraphics[scale=0.33]{Figuras/rio_doce/wfv/RF/RF_WFV_ORIG_RESID_x_PREV.png}
\caption{Dispersão dos resíduos.\\(fonte: o autor)}
\label{fig:doce_RF_WFV_ORIG_RESID_x_PREV}
\end{figure}
\clearpage

No gráfico \ref{fig:doce_CB_WFV_ORIG_RESID_x_TEMPO} é possível observar que no início do período (por volta de janeiro e fevereiro de 2023), os resíduos estão mais dispersos, com muitos valores fora do IQR, tanto na parte superior quanto na inferior. Isso denota que o modelo teve dificuldades em prever corretamente durante essa fase inicial, que pode estar associada a picos de vazão ou condições hidrológicas atípicas. Após março de 2023, os resíduos começam a se concentrar dentro do IQR, indicando que o modelo se estabilizou e começou a prever com mais precisão, com menos erros extremos, ou seja, o CB foi capaz de capturar bem o padrão de vazões ao longo do ano, exceto em eventos de pico no início.

Da mesma forma que o CB, o modelo LR mostra uma maior dispersão dos resíduos no início do período, particularmente nos meses de janeiro e fevereiro de 2023.(figura \ref{fig:doce_LR_WFV_LOG_RESID_x_TEMPO}) Porém, diferentemente do CB, os resíduos não se dispersam, mas ainda há sinais claros de dificuldade em capturar picos elevados de vazão. A partir de março de 2023, os resíduos do LR também se tornam mais estáveis e permanecem concentrados dentro do IQR, indicando uma melhor performance com o passar do tempo e em períodos de amenidades nas vazões.

E, finalmente, o modelo RF.(figura \ref{fig:doce_RF_WFV_ORIG_RESID_x_TEMPO}) O comportamento dos resíduos do RF ao longo do tempo é similar ao dos outros dois modelos anteriores. Há uma maior dispersão no início do ano (janeiro e fevereiro de 2023), mas a partir de março, os resíduos se tornam mais estáveis, com a maioria dos valores dentro do IQR. Indica que o RF também teve dificuldades em capturar os eventos extremos no início do ano, mas apresentou bom desempenho para o restante do período, sugerindo que o RF consegue capturar bem os padrões das vazões após este trecho inicial.

Todos os modelos - CB, LR, RF - apresentaram maior dispersão dos resíduos no trecho inicial do ano de 2023, o que indica que houve dificuldades em prever as vazões nessa fase, provavelmente devido a eventos hidrológicos extremos, como picos extremos de chuva ou vazão. O CB apresenta a maior dispersão e magnitude de \textit{outliers}, enquanto o LR e o RF têm um desempenho ligeiramente melhor, mas igualmente com dificuldades.

Os três modelos enfrentaram desafios semelhantes durante o período inicial do ano (janeiro a fevereiro), mas todos estabilizaram seus resíduos a partir de março, com o RandomForest mostrando uma ligeira vantagem em relação à redução de outliers. O CatBoost, apesar de ser um modelo mais sofisticado, teve mais dificuldades com eventos extremos no início do ano, enquanto a Regressão Linear e o RandomForest apresentaram um desempenho mais consistente, embora ainda com dificuldades em prever eventos extremos de vazão.

Em suma. O comportamento dos resíduos ao longo do tempo indica que, para períodos com picos sazonais, pode ser necessário um refinamento adicional nos modelos ou mesmo refinamento nos dados de entrada para melhor capturar estes eventos.

\begin{figure}[!h]
\centering
\includegraphics[scale=0.33]{Figuras/rio_doce/wfv/CB/CB_WFV_ORIG_RESID_x_TEMPO.png}
\caption{Dispersão dos resíduos ao longo do ano.\\(fonte: o autor)}
\label{fig:doce_CB_WFV_ORIG_RESID_x_TEMPO}
\end{figure}

\begin{figure}[!h]
\centering
\includegraphics[scale=0.33]{Figuras/rio_doce/wfv/LR/LR_WFV_LOG_RESID_x_TEMPO.png}
\caption{Dispersão dos resíduos ao longo do ano.\\(fonte: o autor)}
\label{fig:doce_LR_WFV_LOG_RESID_x_TEMPO}
\end{figure}

\begin{figure}[!h]
\centering
\includegraphics[scale=0.33]{Figuras/rio_doce/wfv/RF/RF_WFV_ORIG_RESID_x_TEMPO.png}
\caption{Dispersão dos resíduos ao longo do ano.\\(fonte: o autor)}
\label{fig:doce_RF_WFV_ORIG_RESID_x_TEMPO}
\end{figure}
\clearpage

A assimetria dos resíduos para o modelo CB - figura \ref{fig:doce_CB_WFV_ORIG_RESID_x_CURVA_NORMAL} - é de $4,16$, o que indica uma distribuição altamente assimétrica à direita - positiva. Isso quer dizer que os erros positivos são mais frequentes ou mais intensos do que os erros negativos. Isso combina com o viés sistemático de todos os modelos em que o PBIAS indicou subestimação - visto em \ref{fig:doce_CB_WFV_ORIG}, \ref{fig:doce_LR_WFV_LOG} e \ref{fig:doce_RF_WFV_ORIG}. Posto que o cálculo do resíduo é ``$observado - previsto$'' e trazendo o gráfico \ref{fig:doce_CB_WFV_ORIG_RESID_x_TEMPO} à vista aqui novamente, percebe-se que esta cauda longa de valores positivos ocorreu principalmente no início do ano, exatamente quando as variações nas vazões foram mais intensas. Este fenômeno também ocorreu com os outros modelos - figuras \ref{fig:doce_LR_WFV_LOG_RESID_x_CURVA_NORMAL} e \ref{fig:doce_RF_WFV_ORIG_RESID_x_CURVA_NORMAL}. Contudo, para o modelo LR com dados log-transformados, a assimetria se mostrou menor - $2,71$ - do que em relação aos modelos não-lineares, restando a assimetria de $3,24$ para o modelo RF.
 
A curva normal - em vermelho - está visivelmente deslocada à direita, devido a assimetria dos dados, para todos os modelos, porém um pouco menos para o modelo LR. Todos os modelos possuem resíduos concentrados, majoritariamente, entre $-200$ e $200$, contudo, para os modelos não-lineares CB e RF, a variação total dos resíduos oscila entre $-400$ e $1000$, ao passo que o modelo LR, mostrando melhor comportamento, oscila entre $-400$ e $800$. Isso tudo mostra que os resíduos não seguem uma distribuição normal, com valores concentrados à direita de zero, o que indica que os modelos não capturaram bem as dinâmicas quando de valores mais elevados de vazão.

\begin{figure}[!h]
\centering
\includegraphics[scale=0.33]{Figuras/rio_doce/wfv/CB/CB_WFV_ORIG_RESID_x_CURVA_NORMAL.png}
\caption{Histograma e curva-normal dos resíduos.\\(fonte: o autor)}
\label{fig:doce_CB_WFV_ORIG_RESID_x_CURVA_NORMAL}
\end{figure}

\begin{figure}[!h]
\centering
\includegraphics[scale=0.33]{Figuras/rio_doce/wfv/LR/LR_WFV_LOG_RESID_x_CURVA_NORMAL.png}
\caption{Histograma e curva-normal dos resíduos.\\(fonte: o autor)}
\label{fig:doce_LR_WFV_LOG_RESID_x_CURVA_NORMAL}
\end{figure}

\begin{figure}[!h]
\centering
\includegraphics[scale=0.33]{Figuras/rio_doce/wfv/RF/RF_WFV_ORIG_RESID_x_CURVA_NORMAL.png}
\caption{Histograma e curva-normal dos resíduos.\\(fonte: o autor)}
\label{fig:doce_RF_WFV_ORIG_RESID_x_CURVA_NORMAL}
\end{figure}
\clearpage

O CB apresenta autocorrelação significativa nos primeiros \textit{lags} - até aproximadamente o \textit{lag} 10 -, com picos consideráveis de correlação positiva, indicando que o modelo não está capturando totalmente as dependências temporais de curto prazo - figura \ref{fig:doce_CB_WFV_ORIG_RESID_ACF}. Este comportamento sugere que, após as previsões imediatas, os resíduos ainda carregam dependências de eventos passados, o que não deveria ocorrer. O LR também mostra autocorrelação nos primeiros \textit{lags}, repetindo o comportamento do CB, embora de menor magnitude quando comparado a este - figura \ref{fig:doce_LR_WFV_LOG_RESID_ACF}. Ainda que o modelo LR também não esteja capturando completamente as dependências de curto prazo, ele o faz de forma um pouco mais eficiente que o CB. Por sua vez, o RF apresenta picos de autocorrelação ainda menores que CB e LR nos primeiros \textit{lags} - figura \ref{fig:doce_RF_WFV_ORIG_RESID_ACF} -, o que indica que o RF consegue capturar melhor as dependências temporais imediatas dos dados. No entanto, ainda persiste alguma autocorrelação, com uma magnitude relativamente baixa, sugerindo que o RF está capturando a dinâmica temporal, se não de forma mais eficiente, com certeza menos deficitária.

Avançando no estudo, a autocorrelação do modelo CB diminui rapidamente após os primeiros \textit{lags} e por volta do \textit{lag} 10, a maioria dos resíduos já se estabiliza dentro do intervalo de confiança - a área azul esmaecida. Isso é um sinal de que, a partir de \textit{lags} maiores, os resíduos são independentes, um comportamento esperado e positivo para previsões de longo prazo. Perceba, no entanto, que tanto o LR quanto o RF apresentam uma dissipação rápida da autocorrelação, diferentemente do CB, e seus resíduos se estabilizam rapidamente dentro do intervalo de confiança após os primeiros \textit{lags}.

Todos os modelos mostraram alguma autocorrelação nos resíduos de curto prazo, indicando que nenhum deles conseguiu eliminar completamente as dependências temporais (de curto prazo). Esse fenômeno é mais acentuado no CB, seguido pela LR e, por fim, pelo RF, que apresenta a menor autocorrelação nos primeiros \textit{lags}. Essa presença presistente de autocorrelação nos resíduos significa que as previsões de curto prazo são influenciadas pelos eventos passados, sugerindo que os modelos não estão capturando eventos imediatamente próximos de maneira ideal.

\begin{figure}[!h]
\centering
\includegraphics[scale=0.33]{Figuras/rio_doce/wfv/CB/CB_WFV_ORIG_RESID_ACF.png}
\caption{Gráfico ACF dos resíduos.\\(fonte: o autor)}
\label{fig:doce_CB_WFV_ORIG_RESID_ACF}
\end{figure}

\begin{figure}[!h]
\centering
\includegraphics[scale=0.33]{Figuras/rio_doce/wfv/LR/LR_WFV_LOG_RESID_ACF.png}
\caption{Gráfico ACF dos resíduos.\\(fonte: o autor)}
\label{fig:doce_LR_WFV_LOG_RESID_ACF}
\end{figure}

\begin{figure}[!h]
\centering
\includegraphics[scale=0.33]{Figuras/rio_doce/wfv/RF/RF_WFV_ORIG_RESID_ACF.png}
\caption{Gráfico ACF dos resíduos.\\(fonte: o autor)}
\label{fig:doce_RF_WFV_ORIG_RESID_ACF}
\end{figure}

Toda essa análise demonstrou aspectos importantes acerca dos modelos. Todos apresentaram uma tendência de subestimar os picos, com o modelo LR tendo uma oscilação levemente menor - de $-400$ a $800$ -, o período do ano que foi complicado para os modelos foi o período inicial do ano, verão, quando as vazões são fortemente impactadas pelas oscilações nas precipitações e o gráfico ACF deixou claro que os resíduos carregam dependência temporal de curto prazo. Isso sugere a necessidade de inclusão de mais variáveis explicativas para lidar com esses eventos extremos, refinamento nos hiperparâmetros - para os modelos não-lineares -, ou mesmo de refinamento nas técnicas de modelagem - isso não pode ser descartado.

\clearpage

\subsection{Rio Grande}

O rio Grande foi o único que a análise foi feita em uma massa reduzida de dados. Conforme já detalhado, apenas dados de setembro de 2020 até dezembro de 2023. Feita esta ressalva, os resultados mostraram-se bons e os modelos foram estáveis quando da análise dos resíduos. Perceba que aqui, para o modelo LR, os melhores resultados foram com os dados escalados MinMax e não os dados log-transformados. O comportamento ruidoso da série de vazão e a quantidade reduzida de registros (cerca de 1200) explicam este comportamento diferenciado.

Iniciando pela métrica KGE, aqui, o RF saiu-se melhor, seguido próximo pelo LR - figuras \ref{fig:grande_RF_WFV_ORIG} e \ref{fig:grande_LR_WFV_ORIG}, respectivamente. Ambos tiveram valores superiores ao CB - figura \ref{fig:grande_CB_WFV_ORIG} -, indicando que capturaram melhor correlação, viés e variabilidade dos dados de vazão ao longo do período.

Sobre o erro médio (MAPE), o RF e o CB apresentaram os menores valores e próximos entre si, ou seja, ambos tiveram comportamento semelhante - valores de $0,123$ e $0,126$, respectivamente. O modelo LR ficou ligeiramente superior, $0,133$, mas de toda forma, isso os coloca com resultados muito próximos, sugerindo que todos os três modelos oferecem previsões de boa qualidade.

O LR tem o menor PBIAS - $-1,99\%$ -, indicando que suas previsões estão mais equilibradas, com menos tendência, neste caso, a subestimar os valores de vazão. O modelos CB e RF também têm PBIAS relativamente baixos - respectivamente $-4,44\%$ e $-2,18\%$ -, mas o CB apresentou um maior valor negativo mais baixo, indicando uma subestimação maior comparado aos outros modelos.

A cobertura empírica do LR é significativamente maior do que a dos outros dois modelos, com $84.38\%$, mostrando que o intervalo de previsão capturou com mais precisão a variabilidade das real presente nas observações. Tanto o CB ($63,56\%$) quanto o RF ($63,01\%$) têm valores de cobertura mais baixos, que mostram, por sua vez, intervalos de previsão foram mais estreitos e não conseguiram captar toda a variabilidade observada nos dados reais.

Em ambos os modelos não-lineares a previsão segue de perto a série observada, mas a subestimação das vazões mais altas é evidente. O intervalo de previsão é estreito em momentos de pico, o que contribui para a baixa cobertura empírica calculada. Os maiores desvios nas previsões ocorrem em períodos de alta vazão (por volta de março e maio de 2023), onde os modelos apresentaram dificuldade em acompanhar as flutuações nas vazões.

O modelo LR consegue capturar razoavelmente bem as flutuações de vazão, mas também apresenta comportamento tendencioso de subestimar os picos mais extremos, da mesma forma que os modelos CB e RF. Contudo, o intervalo de previsão foi o mais amplo dos três modelos, resultando em uma cobertura empírica maior.

O RF foi o modelo que obteve o melhor equilíbrio entre precisão (MAPE), correlação/variabilidade (KGE) e viés (PBIAS), mostrando um desempenho consistente. No entanto, a cobertura empírica baixa sugere que o intervalo de previsão não captura toda a incerteza associada aos dados, mostrando uma característica que pode melhorar. O LR, embora tenha uma MAPE ligeiramente maior, apresentou o melhor valor de PBIAS, denotando que ele tem a menor tendência a viés sistemático entre os modelos. Além disso, sua maior cobertura empírica sugere que o modelo captura melhor a variabilidade dos dados. Finalmente, o CB, ainda que seja um modelo robusto, mostrou desempenho inferior em termos de KGE e PBIAS, o que pode indicar dificuldades em capturar corretamente as variações sazonais e os picos extremos de vazão. Contudo, vale salientar, é um modelo altamente parametrizável e um procedimento de otimização dos hiperparâmetros pode verter melhores resultados.

\begin{figure}[!h]
\centering
\includegraphics[scale=0.33]{Figuras/rio_grande/wfv/CB/CB_WFV_ORIG.png}
\caption{\textit{Walk-Forward Validation} para o modelo CatBoost - CB.\\(fonte: o autor)}
\label{fig:grande_CB_WFV_ORIG}
\end{figure}

\begin{figure}[!h]
\centering
\includegraphics[scale=0.33]{Figuras/rio_grande/wfv/LR/LR_WFV_ORIG.png}
\caption{\textit{Walk-Forward Validation} para o modelo LinearRegression - LR.\\(fonte: o autor)}
\label{fig:grande_LR_WFV_ORIG}
\end{figure}

\begin{figure}[!h]
\centering
\includegraphics[scale=0.33]{Figuras/rio_grande/wfv/RF/RF_WFV_ORIG.png}
\caption{\textit{Walk-Forward Validation} para o modelo RandomForest - RF.\\(fonte: o autor)}
\label{fig:grande_RF_WFV_ORIG}
\end{figure}
\clearpage

A análise dos gráficos de resíduos \textit{versus} valores previstos para os três modelos revela importantes diferenças no comportamento dos resíduos em relação às previsões. De início é possível notar as largas faixas de valores em que os resíduos se concentraram. O CB teve valores de \textit{lower fence} e \textit{upper fence} de $-1328,72$ e $1492,10$, respectivamente, uma amplitude de $2820$, o que lhe conferiu uma prevalência de $89,32\%$ dos resíduos dentro destes limites. De antemão, o menor valor dentre os modelos. Continuando para o modelo LR, tendo $-1628,38$ para o valor da faixa mínima e $1687,81$ para a máxima - amplitude de $3315$ - teve prevalência de 92,60\% dos resíduos. Isso parece realmente bom, mas com uma ampla faixa assim deixa indicado que o modelo oscilou muito nas previsões realizadas, demostrando incerteza. Finalmente, o RF, ficou ``no meio do caminho'' entre os modelos anteriores. Tendo os valores de $-1422,62$ para \textit{lower fence} e $1531,92$ para \textit{upper fence}, perfazendo uma amplitude de $2953$, o modelo teve prevalência de $90,68\%$ dos resíduos dentro destes limites.

Olhando mais atentamente o gráfico do modelo CB - figura \ref{fig:grande_CB_WFV_ORIG_RESID_x_PREV} -, os resíduos estão relativamente concentrados em torno de $0$ para a maioria dos valores previstos. Contudo, à medida que os valores previstos aumentam, os resíduos se tornam mais dispersos, especialmente para previsões acima de $6000 m^3/s$. 

O CB tende a subestimar mais frequentemente as previsões - maior presença de resíduos acima de \textit{upper fence} - para valores elevados de vazão. Existem vários \textit{outliers} positivos e negativos, com valores de resíduos superiores a $2000 m^3/s$ e inferiores a $-1.000 m^3/s$, principalmente para as previsões mais altas, donde pode-se inferir que o CB lidou bem com vazões mais baixas e moderadas, mas apresentou maior dificuldade com picos extremos de vazão, como indicado pela dispersão e pelos \textit{outliers} em previsões de alta magnitude.

O modelo LR também apresentou distribuição de resíduos concentrada em torno de $0$ para valores previstos mais baixos - figura \ref{fig:grande_LR_WFV_ORIG_RESID_x_PREV}. No entanto, a \underline{dispersão} dos resíduos é maior em torno de $0$ e dentro do intervalo mostrado no gráfico, e houve alguns \textit{outliers} para previsões extremamente altas (acima de $7000 m^3/s$). Essa maior dispersão e alguns destes \textit{outliers} extremos indicam dificuldades do modelo em lidar com previsões de vazões de moderadas a extremas.

O RF apresenta uma distribuição de resíduos relativamente mais compacta em torno de $0$ para a maioria dos valores previstos. No entanto, também há uma dispersão crescente à medida que os valores previstos aumentam, semelhante ao CB e ao LR. Um fato importante a se observar, colocando de lado os gráficos \ref{fig:grande_RF_WFV_ORIG_RESID_x_PREV} e \ref{fig:grande_CB_WFV_ORIG_RESID_x_PREV}, em que ambos os comportamentos se assemelham, apesar da faixa levemente larga e da menor prevalência dos resíduos dentro desta, o modelo CB apresentou menos resíduos de valores elevados \underline{abaixo} de \textit{lower fence}, ou seja, o modelo superestimou muito menos as previsões. Porém, acima de \textit{upper fence}, o CB teve mais pontos fora, reforçando o que já tinha sido apresentado antes através da PBIAS, de que este modelo teve tendência sistemática para subestimar as previsões. O modelo RF apresentou uma distribuição acima de \textit{upper fence} e abaixo de \textit{lower fence} mais equilibrada, uma amplitude mediana - quando vista ao lado dos outros modelos -, elevada concentração dos resíduos próximo a $0$, o que faz indicar um comportamento estável holisticamente. Também teve problemas em captar vazões extremas, de certo, mas quando se considera outros parâmetros para mensurar a qualidade, este modelo ficou em um ponto central. E, assim como o CB, ainda pode ser melhorado por otimização de hiperparâmetros.

Quando fala-se em precisão, todos os três modelos - CB, LR e RF - apresentaram resíduos concentrados em torno de $0$ para valores previstos moderados (entre $3000$ e $6000 m^3/s$), indicando que eles conseguem lidar bem com a previsão de vazões nessa faixa, mas para valores extremos, ainda que eventualmente um modelo tenha sido melhor que o outro, foi muito pequena a variação, deixando claro que as previsões extremas foram problemáticas para todos os modelos.

\begin{figure}[!h]
\centering
\includegraphics[scale=0.33]{Figuras/rio_grande/wfv/CB/CB_WFV_ORIG_RESID_x_PREV.png}
\caption{Dispersão dos resíduos.\\(fonte: o autor)}
\label{fig:grande_CB_WFV_ORIG_RESID_x_PREV}
\end{figure}

\begin{figure}[!h]
\centering
\includegraphics[scale=0.33]{Figuras/rio_grande/wfv/LR/LR_WFV_ORIG_RESID_x_PREV.png}
\caption{Dispersão dos resíduos.\\(fonte: o autor)}
\label{fig:grande_LR_WFV_ORIG_RESID_x_PREV}
\end{figure}

\begin{figure}[!h]
\centering
\includegraphics[scale=0.33]{Figuras/rio_grande/wfv/RF/RF_WFV_ORIG_RESID_x_PREV.png}
\caption{Dispersão dos resíduos.\\(fonte: o autor)}
\label{fig:grande_RF_WFV_ORIG_RESID_x_PREV}
\end{figure}
\clearpage

Dispersão dos resíduos no período.

%A análise dos gráficos de dispersão dos resíduos ao longo do tempo para os modelos CatBoost (CB), Regressão Linear (LR) e RandomForest (RF) revela informações cruciais sobre a evolução do desempenho de cada modelo em termos de erro durante o período analisado. A seguir, serão comparados os três modelos com foco em padrões de erro, outliers e estabilidade dos resíduos ao longo do tempo.
%
%### 1. **CatBoost (CB)**
%
%- **Padrão de Erro**: O CB apresenta uma concentração dos resíduos dentro da faixa de ±1.000 \(m^3/s\) para a maior parte do período. No entanto, os primeiros meses (de janeiro a maio de 2023) mostram resíduos mais dispersos, com vários pontos ultrapassando os 2.000 \(m^3/s\). Após esse período, os resíduos se estabilizam um pouco, ficando majoritariamente dentro dos limites.
%- **Outliers**: O CB apresenta vários outliers, especialmente entre março e maio de 2023. Esses outliers indicam que o modelo tem dificuldade em lidar com os picos de vazão durante esse período, resultando em erros elevados.
%- **Estabilidade dos Resíduos**: Após o pico de erro em maio, os resíduos do CB se tornam mais constantes, com variações menores e mais previsíveis. No entanto, a presença de alguns outliers no final do ano sugere que o modelo ainda enfrenta dificuldades em capturar certas variações extremas.
%
%### 2. **Regressão Linear (LR)**
%
%- **Padrão de Erro**: A LR apresenta uma maior dispersão dos resíduos ao longo do tempo em comparação com o CB. Mesmo nos períodos de menor erro, como de julho a outubro de 2023, os resíduos ainda apresentam maior variação, com muitos pontos ultrapassando ±1.000 \(m^3/s\). Isso reflete a limitação da LR em capturar a complexidade dos dados.
%- **Outliers**: A quantidade de outliers é maior no modelo LR do que nos outros dois. Há vários pontos acima de 2.000 \(m^3/s\) e abaixo de -1.000 \(m^3/s\) ao longo de todo o período. Isso sugere que a LR não lida bem com eventos extremos de vazão, resultando em erros significativamente maiores em certos momentos.
%- **Estabilidade dos Resíduos**: A LR parece ter uma menor estabilidade ao longo do tempo, com resíduos variando bastante mesmo em períodos onde o comportamento da série observada é mais estável. A maior variabilidade nos resíduos indica que o modelo linear não captura adequadamente as mudanças dinâmicas nas vazões.
%
%### 3. **RandomForest (RF)**
%
%- **Padrão de Erro**: O RF apresenta uma dispersão dos resíduos semelhante ao CB, mas com uma menor quantidade de outliers. A maior parte dos resíduos fica dentro da faixa de ±1.000 \(m^3/s\), com algumas exceções nos meses de março e abril de 2023, onde o modelo parece encontrar mais dificuldade em prever corretamente as vazões.
%- **Outliers**: O RF tem menos outliers do que a LR e o CB, o que sugere que ele é mais robusto na captura das variações de vazão ao longo do tempo. No entanto, os outliers que estão presentes ocorrem durante os períodos de maior instabilidade nas vazões.
%- **Estabilidade dos Resíduos**: O RF mostra uma maior estabilidade dos resíduos ao longo do tempo, com menor dispersão e uma tendência clara de erros mais baixos em comparação com a LR. No entanto, ainda há alguns picos de erro em períodos de maior variação nas vazões, especialmente nos primeiros meses de 2023.
%
%### **Comparação Geral**
%
%1. **Padrão Geral dos Erros**:
%- O **CB** e o **RF** mostram um padrão de erro mais estável ao longo do tempo, com a maioria dos resíduos concentrados dentro de uma faixa controlada (±1.000 \(m^3/s\)). Ambos apresentam períodos de erro maior, especialmente no início do ano, mas estabilizam conforme o tempo avança.
%- A **LR**, por outro lado, mostra maior variabilidade nos resíduos, sugerindo que o modelo não lida tão bem com as flutuações nas vazões, independentemente do período. A maior quantidade de resíduos dispersos e outliers indica que o modelo linear tem dificuldades em capturar a não linearidade dos dados.
%
%2. **Outliers**:
%- A **LR** tem a maior quantidade de outliers, especialmente no início do período, onde os erros são maiores e mais frequentes. Isso reflete a inadequação do modelo linear para prever vazões em cenários de maior complexidade.
%- O **CB** também apresenta outliers consideráveis, mas em menor quantidade do que a LR. O RF, por sua vez, tem o menor número de outliers, indicando que ele é o modelo mais robusto para lidar com as variações extremas, embora ainda enfrente desafios nos picos de vazão.
%
%3. **Estabilidade dos Resíduos**:
%- O **RF** se destaca como o modelo mais estável ao longo do tempo, com uma menor variabilidade dos resíduos e menos outliers em comparação aos outros dois. Isso sugere que o RF é mais adequado para lidar com a variação temporal das vazões, mantendo uma previsão mais confiável e consistente.
%- O **CB** também tem uma boa estabilidade, mas apresenta mais flutuações nos resíduos, especialmente no início do período. Isso indica que o CB pode precisar de mais ajustes ou refinamento em eventos de maior magnitude.
%- A **LR** é claramente o modelo menos estável, com uma variabilidade considerável dos resíduos ao longo do tempo e maior quantidade de erros em períodos críticos. A incapacidade da LR de capturar a complexidade dos dados de vazão é evidente na distribuição mais dispersa dos resíduos.
%
%### **Conclusões Importantes**
%
%1. **CatBoost (CB)**: O CB lida bem com previsões de vazões moderadas e mais constantes, mas apresenta dificuldades em períodos de alta variação, especialmente no início do período. Apesar disso, ele é um modelo competitivo, apresentando um desempenho razoável ao longo do tempo com menor número de outliers do que a LR.
%
%2. **Regressão Linear (LR)**: O desempenho da LR é o mais inconsistente dos três modelos. A grande variabilidade dos resíduos e o número elevado de outliers indicam que o modelo linear é inadequado para captar a complexidade e a não linearidade dos dados de vazão, tornando-o uma escolha menos confiável para previsões precisas.
%
%3. **RandomForest (RF)**: O RF é o modelo mais robusto, com menor quantidade de outliers e uma melhor estabilidade dos resíduos ao longo do tempo. Ele se destaca por capturar as flutuações de vazão de maneira mais precisa e consistente, tornando-o o modelo mais confiável entre os três analisados.
%
%No geral, o **RF** é a melhor escolha em termos de estabilidade e precisão ao longo do tempo, seguido pelo **CB**, que também tem um desempenho aceitável, mas com mais flutuações. A **LR**, por sua vez, não é recomendada para esse tipo de previsão, devido à sua alta variabilidade de erros e grande quantidade de outliers.

\begin{figure}[!h]
\centering
\includegraphics[scale=0.33]{Figuras/rio_grande/wfv/CB/CB_WFV_ORIG_RESID_x_TEMPO.png}
\caption{Dispersão dos resíduos ao longo do ano.\\(fonte: o autor)}
\label{fig:grande_CB_WFV_ORIG_RESID_x_TEMPO}
\end{figure}

\begin{figure}[!h]
\centering
\includegraphics[scale=0.33]{Figuras/rio_grande/wfv/LR/LR_WFV_ORIG_RESID_x_TEMPO.png}
\caption{Dispersão dos resíduos ao longo do ano.\\(fonte: o autor)}
\label{fig:grande_LR_WFV_ORIG_RESID_x_TEMPO}
\end{figure}

\begin{figure}[!h]
\centering
\includegraphics[scale=0.33]{Figuras/rio_grande/wfv/RF/RF_WFV_ORIG_RESID_x_TEMPO.png}
\caption{Dispersão dos resíduos ao longo do ano.\\(fonte: o autor)}
\label{fig:grande_RF_WFV_ORIG_RESID_x_TEMPO}
\end{figure}
\clearpage

Histograma e curva-normal.

%Ao analisar a distribuição dos resíduos para os três modelos (CatBoost - CB, Regressão Linear - LR e RandomForest - RF) no Rio Grande, podemos observar alguns padrões relevantes na forma como cada modelo lida com os erros de previsão. A seguir, faço a análise e comparação com base nas distribuições apresentadas.
%
%### 1. **CatBoost (CB)**
%- **Assimetria**: O gráfico do CB mostra uma assimetria de 1,3, indicando uma distribuição levemente assimétrica para a direita. Isso sugere que, embora a maior parte dos resíduos esteja concentrada perto de zero, há uma quantidade considerável de erros positivos (previsões subestimadas), especialmente na faixa de 2.000 a 3.000 \(m^3/s\).
%- **Distribuição**: A curva normal ajustada demonstra que os resíduos seguem uma forma próxima da normalidade, porém com uma cauda longa à direita. Isso indica que o CB tende a cometer alguns erros maiores nas previsões de altas vazões.
%- **Outliers**: Existem poucos outliers acima de 3.000 \(m^3/s\), o que mostra que o modelo tem alguma dificuldade em prever corretamente os valores de vazão mais extremos.
%
%### 2. **Regressão Linear (LR)**
%- **Assimetria**: A assimetria de 0,97 indica que a distribuição dos resíduos no modelo LR é quase simétrica, com um leve viés para a direita. Isso significa que os erros estão razoavelmente bem distribuídos em torno de zero, com uma leve tendência de subestimativas.
%- **Distribuição**: A distribuição dos resíduos segue mais de perto a curva normal ajustada, com uma forma mais equilibrada e uma dispersão menor. No entanto, ainda há resíduos significativos acima de 1.000 \(m^3/s\), o que sugere que a LR pode não capturar bem as variações maiores nas vazões.
%- **Outliers**: Existem menos outliers extremos em comparação ao CB, mas o número de resíduos mais distantes da média ainda é considerável. Isso reflete a limitação de um modelo linear em capturar a complexidade dos dados de vazão.
%
%### 3. **RandomForest (RF)**
%- **Assimetria**: O modelo RF apresenta a menor assimetria, com um valor de 0,82, o que indica uma distribuição de resíduos mais simétrica. Isso sugere que o RF lida melhor com os erros, distribuindo-os de maneira mais uniforme em torno de zero.
%- **Distribuição**: A distribuição dos resíduos segue de forma mais precisa a curva normal ajustada em comparação com CB e LR, o que indica que o modelo RF tem uma maior capacidade de gerar previsões com erros menores e menos dispersos.
%- **Outliers**: O RF tem o menor número de outliers em comparação com os outros dois modelos, o que sugere que é mais robusto ao lidar com valores extremos de vazão.
%
%### **Comparação Geral**
%
%1. **Assimetria**:
%- O **CB** apresenta a maior assimetria (1,3), indicando que ele tende a subestimar as vazões em maior frequência. Isso é visível pela cauda longa à direita da distribuição.
%- A **LR** tem uma assimetria ligeiramente menor (0,97), o que indica uma distribuição de resíduos um pouco mais equilibrada, mas ainda com uma tendência para erros positivos.
%- O **RF**, com uma assimetria de 0,82, é o modelo mais simétrico, o que significa que seus erros são distribuídos de maneira mais equitativa entre superestimações e subestimações.
%
%2. **Distribuição**:
%- O **RF** tem a distribuição de resíduos mais próxima de uma distribuição normal, o que indica que é o modelo mais previsível e consistente em suas previsões.
%- O **CB** e a **LR** mostram distribuições com caudas mais longas, especialmente para previsões superestimadas, indicando que ambos os modelos têm dificuldade em lidar com valores extremos de vazão.
%
%3. **Outliers**:
%- O **CB** apresenta mais outliers em valores altos de vazão, sugerindo que o modelo tem uma dificuldade maior em lidar com picos extremos de vazão.
%- A **LR** apresenta menos outliers em comparação ao CB, mas ainda assim sua capacidade de lidar com os extremos é limitada.
%- O **RF**, novamente, se destaca com o menor número de outliers, mostrando que ele é o modelo mais robusto em prever cenários mais extremos.
%
%### **Conclusões Importantes**
%
%- **CatBoost**: Embora seja um modelo poderoso, o CB parece subestimar frequentemente as vazões mais altas, resultando em uma assimetria de 1,3. A presença de outliers na cauda direita da distribuição sugere que o CB tem mais dificuldade em prever corretamente eventos extremos de alta vazão.
%
%- **Regressão Linear**: A **LR** mostra uma distribuição mais equilibrada dos resíduos, com uma assimetria próxima de zero. No entanto, o modelo linear ainda não consegue capturar adequadamente as dinâmicas complexas da série de vazões, refletido pela presença de resíduos elevados e uma quantidade considerável de outliers.
%
%- **RandomForest**: O **RF** é o modelo que apresenta a melhor distribuição dos resíduos, com a menor assimetria e o menor número de outliers. Isso faz do RF o modelo mais robusto para lidar com as variáveis não lineares e as variações extremas de vazão, gerando previsões mais consistentes e com menor dispersão dos erros.
%
%No geral, o **RF** se destaca como o modelo mais equilibrado e confiável para previsões de vazão no Rio Grande, seguido pelo **CB**, que ainda pode ser melhorado para lidar com vazões mais altas. A **LR** mostra-se a menos eficaz, devido à sua limitação em capturar a complexidade dos dados de vazão.

\begin{figure}[!h]
\centering
\includegraphics[scale=0.33]{Figuras/rio_grande/wfv/CB/CB_WFV_ORIG_RESID_x_CURVA_NORMAL.png}
\caption{Histograma e curva-normal dos resíduos.\\(fonte: o autor)}
\label{fig:grande_CB_WFV_ORIG_RESID_x_CURVA_NORMAL}
\end{figure}

\begin{figure}[!h]
\centering
\includegraphics[scale=0.33]{Figuras/rio_grande/wfv/LR/LR_WFV_ORIG_RESID_x_CURVA_NORMAL.png}
\caption{Histograma e curva-normal dos resíduos.\\(fonte: o autor)}
\label{fig:grande_LR_WFV_ORIG_RESID_x_CURVA_NORMAL}
\end{figure}

\begin{figure}[!h]
\centering
\includegraphics[scale=0.33]{Figuras/rio_grande/wfv/RF/RF_WFV_ORIG_RESID_x_CURVA_NORMAL.png}
\caption{Histograma e curva-normal dos resíduos.\\(fonte: o autor)}
\label{fig:grande_RF_WFV_ORIG_RESID_x_CURVA_NORMAL}
\end{figure}
\clearpage

ACF

%Vamos realizar uma análise comparativa e interpretação dos gráficos apresentados para os três modelos de previsão (RandomForest, Regressão Linear e CatBoost) com base na função de autocorrelação dos resíduos.
%
%### RandomForest (RF)
%O gráfico ACF do RF apresenta um padrão onde os primeiros 10 lags possuem autocorrelação significativa e positiva. Isso pode indicar a presença de estrutura nos resíduos que não foi completamente capturada pelo modelo, sugerindo que o modelo ainda possui dependências temporais não modeladas adequadamente. A partir do lag 15, as correlações entram na faixa de confiança (área azul), indicando uma melhora na independência dos resíduos, mas a flutuação entre autocorrelações positivas e negativas dentro da faixa ainda é perceptível, sugerindo resíduos levemente estruturados ao longo do tempo.
%
%### Regressão Linear (LR)
%O modelo de Regressão Linear, por sua vez, mostra uma característica similar ao RF em relação aos primeiros lags: os primeiros 10 lags também apresentam autocorrelação positiva significativa. A dependência temporal não modelada adequadamente é mais evidente nos primeiros lags, mas assim como no RF, à medida que avançamos no tempo, a autocorrelação diminui, embora existam flutuações. Há uma melhoria considerável após o lag 20, onde os resíduos começam a oscilar mais perto de zero e dentro da faixa de confiança, sugerindo que o modelo, apesar de simples, está conseguindo capturar alguns padrões, mas há espaço para melhorias na modelagem temporal.
%
%### CatBoost (CB)
%O gráfico ACF para o CatBoost também exibe um padrão semelhante nos primeiros lags, com autocorrelações significativas nos primeiros 10-15 lags, indicando que o modelo ainda tem algumas dependências temporais não modeladas. No entanto, comparado ao RF e à Regressão Linear, o CatBoost parece ter uma recuperação mais rápida, com a maior parte dos resíduos atingindo a faixa de confiança mais cedo. Isso sugere que o modelo tem uma ligeira vantagem sobre os outros em capturar a estrutura dos dados e modelar as dependências temporais, embora ainda apresente alguma autocorrelação nos primeiros lags.
%
%### Conclusões Gerais:
%- **Dependência Temporal:** Todos os modelos apresentam algum nível de autocorrelação significativa nos primeiros lags, o que sugere que nenhum dos modelos conseguiu modelar completamente as dependências temporais presentes nos dados. Isso indica que poderia haver melhorias ao incorporar algum mecanismo para lidar com a sazonalidade ou dependência temporal diretamente (por exemplo, modelos autorregressivos ou lag features mais robustas).
%
%- **CatBoost:** O CatBoost se destacou por diminuir as autocorrelações de forma mais eficaz do que os outros dois modelos. Isso está em linha com sua arquitetura que é mais adaptável e tem maior capacidade de modelagem em relação a variáveis complexas, capturando melhor padrões não lineares e estruturais nos dados.
%
%- **Regressão Linear:** Por ser um modelo simples e linear, a regressão linear é o modelo que mais sofre com as dependências temporais e estruturais nos resíduos. Sua incapacidade de capturar padrões não lineares ou interações mais complexas reflete na quantidade de resíduos que mostram correlação nos primeiros lags.
%
%- **RandomForest:** Apesar de ser um modelo baseado em árvores, que teoricamente é mais flexível que a Regressão Linear, o RandomForest ainda apresenta significativa autocorrelação nos resíduos, especialmente nos primeiros lags. Isso pode ser reflexo de sua natureza não-paramétrica e, potencialmente, da falta de features temporais ou lags adequados no modelo.
%
%Em suma, nenhum dos modelos conseguiu eliminar completamente a dependência temporal dos resíduos, mas o **CatBoost** apresentou uma performance superior no que tange à redução dessas dependências. Isso sugere que para melhorar a precisão dos modelos, seria interessante explorar a incorporação de características temporais adicionais ou técnicas como a de modelagem autorregressiva.

\begin{figure}[!h]
\centering
\includegraphics[scale=0.33]{Figuras/rio_grande/wfv/CB/CB_WFV_ORIG_RESID_ACF.png}
\caption{Gráfico ACF dos resíduos.\\(fonte: o autor)}
\label{fig:grande_CB_WFV_ORIG_RESID_ACF}
\end{figure}

\begin{figure}[!h]
\centering
\includegraphics[scale=0.33]{Figuras/rio_grande/wfv/LR/LR_WFV_ORIG_RESID_ACF.png}
\caption{Gráfico ACF dos resíduos.\\(fonte: o autor)}
\label{fig:grande_LR_WFV_ORIG_RESID_ACF}
\end{figure}

\begin{figure}[!h]
\centering
\includegraphics[scale=0.33]{Figuras/rio_grande/wfv/RF/RF_WFV_ORIG_RESID_ACF.png}
\caption{Gráfico ACF dos resíduos.\\(fonte: o autor)}
\label{fig:grande_RF_WFV_ORIG_RESID_ACF}
\end{figure}
\clearpage

\subsection{Rio São Francisco}

\section{Importância das variáveis}
%Analisar a importância das variáveis contínuas e categóricas na previsão (feature importance).

\section{Discuss\~ao dos resultados}
%Interpretar os resultados e discutir as limitações. Se possível, comparar com estudos anteriores.