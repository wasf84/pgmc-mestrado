\chapter{INTRODU\c{C}\~AO}  %%Nesta linha, dentro de { }, digita-se em CAIXA ALTA, como apresentado aqui

% \begin{itemize}
	%     \item \textbf{use o comando cite para fazer a citação so com o número da referencia \cite{rayyan-33388112}}
	
	%     \item \textbf{use o comando citet para fazer a citação com o nome do autor e o numero da referencia \citet{rayyan-33388114}}
	% \end{itemize}

% \begin{figure}[htb!]
	%     \centering
	%     \includegraphics[width=0.5\linewidth]{Figuras/google.png}
	%     \caption{Caption}
	%     \label{fig:enter-label}
	% \end{figure}

% \begin{figure}[htb!]
	%     \centering
	%     \includegraphics[scale=0.3]{Figuras/google.png}
	%     \caption{Caption}
	%     \label{fig:enter-label}
	% \end{figure}

A gestão dos recursos hídricos desempenha um papel crucial nas políticas públicas. Nos âmbitos socioeconômico, cultural e de saúde pública, conhecer a dinâmica dos recursos hídricos e entender como fatores externos impactam seu comportamento é de grande importância para os administradores públicos. A compreensão desses aspectos permite uma melhor tomada de decisões, garantindo a sustentabilidade dos recursos, a segurança hídrica e o bem-estar da população.

Neste sentido, prever a vazão de rios é um componente essencial na gestão de recursos hídricos, operação de reservatórios e mitigação de desastres naturais, especialmente em regiões onde a hidroelectricidade desempenha um papel crucial na matriz energética, como é o caso do Brasil\todo[color=blue,textcolor=white]{tem referencia para esta parte? seria bom colocar alguma coisa}. De acordo com o Balanço Energético Nacional de 2023, ano-base 2022, divulgado pelo Ministério de Minas e Energia, esta matriz energética representa cerca de 64\% da oferta interna total de geração de energia elétrica \cite{epe2023ben}. Desta forma, a previsão da vazão dos rios que abastecem os reservatórios das hidrelétricas tem importância no impacto econômico que uma usina em baixa capacidade de geração pode causar.

Em uma perspectiva mais direcionada à população, os rios abastecem represas e açudes que fornecem água potável para consumo humano e animal, além de irrigar a lavoura \todo[color=blue, textcolor=white]{referencia?}. Não apenas os rios, mas também a chuva têm impacto significativo nesse cenário. Uma análise criteriosa da previsibilidade da vazão dos rios nos dias seguintes permite ao poder público, por exemplo, reduzir ou até mesmo suspender outorgas para retirada de água, visando o bem-estar populacional. Além disso, essa análise pode auxiliar no planejamento de regimes de racionamento. Basta lembrarmos do ano de 2015, quando noticiava-se o 'uso do volume morto' \todo[color=blue, textcolor=white]{aspas no latex é feito assim ''exemplo de aspas'', deixe as aspas simples para as variáveis utilizadas nos eu modelo} na Cantareira, no estado de São Paulo, pois a estiagem fora além do previsto e o abastecimento de cidades, da cidade de São Paulo propriamente, foram severamente afetados.\cite{g1_cantareira_2015}

E quando se fala em bem-estar populacional, não podemos deixar de considerar os eventos climáticos extremos. 

%No Brasil, a ocorrência de inundações em áreas urbanas tem-se intensificado devido a diversos fatores, dentre eles a urbanização acelerada e a ocupação inadequada do solo. Além da impermeabilização dos solos, falta infraestrutura nos municípios e o desmatamento ciliar também contribui para as inundações, favorecendo o aumento do volume de vazão e a velocidade de propagação da onda de inundação em regiões onde os eventos hidrológicos constituem um risco de desastre natural (ref. CEMADEN).

Basta lembrar dos últimos desastres ocorridos nos estados de Minas Gerais, Rio de Janeiro, Paraná, São Paulo e Bahia que trouxeram não só perda econômica como também perda de vidas humanas. \cite{bbc_chuvas_bahia_2024} \cite{cnn_temporais_mg_2024} \cite{g1_temporal_es_2024} \cite{g1_temporal_petropolis_2022}

Especificamente no estado de Minas Gerais há lacunas de conhecimento sobre os processos hidrológicos das bacias hidrográficas

\section{Contextualização}
%Introduzir o tema, contextualizando a área de estudo e a relevância da pesquisa.

\section{Justificativa}
%Explicar porque a pesquisa é importante e quais são contribuições.

\section{Problema de pesquisa}
%Definir claramente o problema que a pesquisa busca resolver.

\section{Objetivos}

\todo[inline, color=blue, textcolor=white]{Separa objetivo e objetivo específico em seções}
O trabalho tem como escopo a modelagem de recursos hídricos baseada em dados históricos de vazão, contrastando com os modelos físicos tradicionalmente aplicados na área, como o modelo SMAP (Soil Moisture Accounting Procedure) e o SWAT (Soil and Water Assessment Tool). Ao invés de utilizar as abordagens físicas que simulam processos hidrológicos baseados em equações diferenciais e características fisiográficas das bacias, a proposta é aplicar métodos de modelagem orientados a dados para prever séries temporais univariadas de vazão. Essas previsões serão realizadas com o suporte de variáveis exógenas, como precipitação, e variáveis categóricas, a serem melhor descritas a posteri, que ajudam a capturar a sazonalidade e outros padrões importantes do regime de águas.

\todo[inline, color=blue, textcolor=white]{O objetivo específico deve vir em tópicos e em uma subseção do objetivo}
O objetivo mais específico é desenvolver um modelo de previsão eficiente e preciso que possa ser utilizado para auxiliar na gestão dos recursos hídricos em Minas Gerais. Em uma etapa posterior, pretende-se criar uma aplicação web para disponibilizar essas previsões ao Instituto Mineiro de Gestão das Águas (IGAM \cite{igam_site_2024}), permitindo que as informações estejam acessíveis para planejamento e tomada de decisões estratégicas.

Este trabalho se insere como um componente fundamental de um sistema gerencial maior, que visa aprimorar o planejamento e a gestão dos recursos hídricos do estado de Minas Gerais. Ao integrar previsões baseadas em dados históricos, o modelo permitirá uma melhor alocação dos recursos, suporte em períodos críticos, como secas e enchentes, e uma gestão mais eficiente das bacias hidrográficas do estado. \todo[color=blue,textcolor=white]{esta parte me parece mais uma descrição do trabalho do que um objetivo propriamente dito}

\section{Estrutura da Dissertação}
\todo[inline, color=blue, textcolor=white]{não é muito usual, alguns membros da banca podem não gostar, verificar com o léo se cabe esta parte na sua dissertação}
Resumo breve da organização dos capítulos.