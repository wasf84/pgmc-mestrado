\chapter{METODOLOGIA}
\section{Descrição da Área de Estudo}
Detalhar a localização e características da minha área de estudo.

\section{Dados Utilizados}
\todo[inline]{FAZER IMAGEM DE ONDE ESTÃO AS ESTAÇÕES}

Os dados de precipitação e vazão utilizados nesta pesquisa foram obtidos a partir do site da Agência Nacional de Águas e Saneamento Básico (ANA), por meio da biblioteca HydroBR \cite{carvalho2020hydrobr}. Esta biblioteca permitiu a listagem de todas as estações hidrométricas disponíveis, como, por exemplo, as estações convencionais de medição de vazão. Após a identificação e seleção das estações de interesse, cujos códigos estavam disponíveis na base de dados da ANA, desenvolveu-se um conjunto de funções para automatizar o processo de extração. Essas funções permitiram o \textit{download} dos dados referentes ao período especificado diretamente do \textit{webservice} fornecido pela ANA.

O período de dados analisado compreende \textbf{de 1º de janeiro de 2013 a 31 de dezembro de 2023}, totalizando 11 anos completos.

Foram utilizadas \textbf{séries temporais diárias} de precipitação e vazão. As colunas correspondentes às datas foram formatadas como `\textit{datetime}', enquanto os dados de precipitação e vazão foram representados como valores de ponto flutuante (`\textit{float}'). Embora a frequência diária tenha sido adotada, é importante destacar que nem todas as séries temporais estavam originalmente nesse formato. Foi necessário lidar com quebra na continuidade das datas e com dados ausentes. Estes aspectos serão discutidos em detalhes em seções subsequentes.

Os dados de precipitação e vazão obtidos do site da ANA já estavam ajustados nas escalas padrão utilizadas em estudos hidrológicos. A precipitação foi fornecida em milímetros por dia (mm/dia), refletindo a quantidade de chuva que cai sobre uma unidade de área em um período de 24 horas e as vazões, por sua vez, foram disponibilizadas em metros cúbicos por segundo (m³/s), indicando o volume de água que passa por uma seção transversal do rio a cada segundo. Em algumas estações, foram observados valores extremamente elevados para determinados dias, tanto nas séries de precipitação quanto nas de vazão, os quais podem ser considerados \textit{outliers}. Em relação aos dados de vazão, verificou-se a ocorrência de valores nulos (vazão igual a 0), o que indicaria a interrupção completa do fluxo do rio. Esse fenômeno, no entanto, não faz sentido, considerando que não há registro de eventos de seca tão severos nos rios analisados, conforme constatado na revisão bibliográfica e em fontes jornalísticas. Apesar destas anomalias, os dados não foram descartados, pois tanto os registros de vazão quanto os de precipitação utilizados nesta pesquisa foram considerados consistidos pela ANA, ou seja, foram medidos e validados pela agência. O presente trabalho não questionou a veracidade dos dados; eles foram utilizados conforme disponibilizados pela ANA.

É relevante destacar que a consulta prévia ao sistema \textit{on-line} da ANA foi essencial, pois frequentemente selecionavam-se códigos de estação que, ao final, não possuíam dados para o período especificado ou apresentavam códigos alterados na base de dados, sendo retornados como ``inexistentes''. Quando um código de estação não retornava resultados na consulta ao sistema, foi necessário utilizar o sistema gentilmente cedido pela Rhama Analysis para verificar se o código da estação havia sido modificado. Nos casos em que se constatava a alteração, o novo código foi adotado, enquanto o código anteriormente informado como inexistente foi descartado.

Em cada rio analisado, a estação alvo, com a vazão que se pretendia prever, foram destacadas em itálico para ficar claro ao leitor como identificá-las.

A distinção entre estação convencional e telemétrica deve-se a esta ter informações a cada quinze minutos, a cada trinta minutos ou ser do tipo horária. Onde ocorreu de ter informações tão granuladas assim, para a precipitação foi feito o somatório para um dia e a vazão foi a média de um dia.

Por fim, é importante destacar a existência de estações híbridas, classificadas como ``pluviométricas/fluviométricas''. Em alguns casos, o código da estação pode indicar que se trata de uma estação de vazão (com códigos iniciados em 5 ou 6, por exemplo), mas que também possui informações de precipitação. O inverso também ocorre, onde códigos indicam estações pluviométricas (com códigos iniciados em 016 ou 019, por exemplo) que, no entanto, contêm dados de vazão. Para garantir a consistência com a nomenclatura utilizada pela ANA, manteve-se a classificação original das estações, mesmo que estas contenham apenas dados de precipitação ou vazão.

Para facilitar a visualização, as estações de vazão e precipitação utilizadas no trabalho são apresentadas abaixo. \\

\begin{table}[!h]
\centering \small
\caption{Estações usadas no rio Jequitinhonha}
\begin{tabular}{|c|c|c|c|c|} \hline 
%\multicolumn{5}{|c|}{\textbf{Rio Jequitinhonha}}\\ \hline \hline
%\multicolumn{5}{|c|}{\textbf{Convencionais}}\\ \hline 
%\multicolumn{5}{|c|}{\textbf{Fluviométricas}}\\ \hline
%\textbf{Código}&  \textbf{Nome}& \textbf{Município}& \textbf{Latitude}& \textbf{Longitude}\\\hline
%54780000& JACINTO& JACINTO& -16,1358& -40,3061\\\hline
%\multicolumn{5}{|c|}{}\\\hline
\multicolumn{5}{|c|}{\textbf{Telemétricas}}\\\hline
\multicolumn{5}{|c|}{\textbf{Pluviométricas/Fluviométricas}}\\\hline
\textbf{Código}   & \textbf{Nome}                                 & \textbf{Município}       & \textbf{Latitude} & \textbf{Longitude}\\\hline
\textit{54790000} & \textit{\makecell{UHE ITAPEBI \\ MONTANTE 1}} & \textit{SALTO DA DIVISA} & \textit{-16,08}   & \textit{-40,0521}\\\hline
01640000          & JACINTO                                       & JACINTO                  & -16,1386          & -40,2903\\\hline
\end{tabular}
\label{tab:estacoes_jequitinhonha}
\end{table}

\begin{table}[!h]
\centering \small
\caption{Estações usadas no rio Doce}
\begin{tabular}{|c|c|c|c|c|} \hline 
%\multicolumn{5}{|c|}{\textbf{Rio Doce}}\\ \hline \hline
\multicolumn{5}{|c|}{\textbf{Convencionais}}\\ \hline 
\multicolumn{5}{|c|}{\textbf{Pluviométricas}}\\ \hline
\textbf{Código} & \textbf{Nome}                               & \textbf{Município} & \textbf{Latitude} & \textbf{Longitude}\\\hline
01941010        & \makecell{SÃO SEBASTIÃO \\ DA ENCRUZILHADA} & AIMORÉS            & -19,4925          & -41,1617 \\\hline
01941004        & \makecell{RESPLENDOR - JUSANTE}             & RESPLENDOR         & -19,3431          & -41,2461 \\\hline
01941006        & \makecell{ASSARAI - MONTANTE}               & POCRANE            & -19,5947          & -41,4581 \\\hline
%\multicolumn{5}{|c|}{\textbf{Fluviométricas}}\\ \hline
%\textbf{Código}&  \textbf{Nome}& \textbf{Município}& \textbf{Latitude}& \textbf{Longitude}\\\hline
%56990000 & \makecell{SÃO SEBASTIÃO \\ DA ENCRUZILHADA} & AIMORÉS & -19,4925 & -41,1617 \\\hline
%56989900 & \makecell{BARRA DO CAPIM} & AIMORÉS & -19,4903 & -41,2033 \\\hline
%56989400 & \makecell{ASSARAI - MONTANTE} & POCRANE & -19,5947 & -41,4581 \\\hline
\multicolumn{5}{|c|}{}\\\hline
\multicolumn{5}{|c|}{\textbf{Telemétricas}}\\\hline
%\multicolumn{5}{|c|}{\textbf{Fluviométricas}}\\\hline
%\textbf{Código}& \textbf{Nome}& \textbf{Município}& \textbf{Latitude}& \textbf{Longitude}\\\hline
%56990850 & \makecell{UHE AIMORÉS \\ BARRAMENTO} & AIMORÉS & -19,4564 & -41,0954 \\\hline
%56990005 & \makecell{UHE AIMORÉS \\ RIO MANHUAÇU} & AIMORÉS & -19,4917 & -41,1614 \\\hline 
\multicolumn{5}{|c|}{\textbf{Pluviométricas/Fluviométricas}}\\\hline
\textbf{Código}   & \textbf{Nome}                          & \textbf{Município} & \textbf{Latitude} & \textbf{Longitude}\\\hline
56990005          & \makecell{UHE AIMORÉS \\ RIO MANHUAÇU} & AIMORÉS            & -19,4917          & -41,1614 \\\hline 
\textit{56994500} & \textit{\makecell{COLATINA PONTE}}     & \textit{COLATINA}  & \textit{-19,5333} & \textit{-40,6297} \\\hline
\end{tabular}
\label{tab:estacoes_rio_doce}
\end{table}

\begin{table}[!h]
\centering \small
\caption{Estações usadas no rio Grande}
\begin{tabular}{|c|c|c|c|c|} \hline 
%\multicolumn{5}{|c|}{\textbf{Rio Grande}}\\ \hline \hline
%\multicolumn{5}{|c|}{\textbf{Convencionais}}\\ \hline 
%\multicolumn{5}{|c|}{\textbf{Pluviométricas}}\\ \hline
%\textbf{Código}&  \textbf{Nome}& \textbf{Município}& \textbf{Latitude}& \textbf{Longitude}\\\hline
%01950006 & \makecell{UHE ÁGUA VERMELHA \\ BARRAMENTO} & OUROESTE & -19,8619 & -50,3456 \\\hline
%02050001 & \makecell{SANTA ALBERTINA} & SANTA ALBERTINA & -20,0333 & -50,7333 \\\hline
%\multicolumn{5}{|c|}{}\\\hline
\multicolumn{5}{|c|}{\textbf{Telemétricas}}\\\hline
\multicolumn{5}{|c|}{\textbf{Fluviométricas}}\\\hline
\textbf{Código}   & \textbf{Nome}                                       & \textbf{Município}     & \textbf{Latitude} & \textbf{Longitude}\\\hline
\textit{62020080} & \textit{\makecell{UHE ILHA SOLTEIRA \\ BARRAMENTO}} & \textit{ILHA SOLTEIRA} & \textit{-20,3797} & \textit{-51,3686} \\\hline 
\multicolumn{5}{|c|}{\textbf{Pluviométricas/Fluviométricas}}\\ \hline
\textbf{Código} & \textbf{Nome}                              & \textbf{Município} & \textbf{Latitude} & \textbf{Longitude}\\\hline
61998080        & \makecell{UHE ÁGUA VERMELHA \\ BARRAMENTO} & OUROESTE           & -19,8628          & -50,3475 \\\hline
\end{tabular}
\label{tab:estacoes_rio_grande}
\end{table}

\begin{table}[!h]
\centering \small
\caption{Estações usadas no rio São Francisco}
\begin{tabular}{|c|c|c|c|c|} \hline 
%\multicolumn{5}{|c|}{\textbf{Rio São Francisco}}\\ \hline \hline
\multicolumn{5}{|c|}{\textbf{Convencionais}}\\ \hline 
\multicolumn{5}{|c|}{\textbf{Fluviométricas}}\\ \hline
\textbf{Código}   & \textbf{Nome}                                 & \textbf{Município}                            & \textbf{Latitude} & \textbf{Longitude}\\\hline
\textit{44290002} & \textit{\makecell{PEDRAS DE MARIA\\ DA CRUZ}} & \textit{\makecell{PEDRAS DE MARIA\\ DA CRUZ}} & \textit{-15,6011} & \textit{-44,3967} \\\hline
%44250000 & \makecell{USINA DO PANDEIROS\\ MONTANTE} & JANUÁRIA & -15,4831 & -44,7681 \\\hline
\multicolumn{5}{|c|}{\textbf{Pluviométricas/Fluviométricas}}\\ \hline
\textbf{Código} & \textbf{Nome}                            & \textbf{Município} & \textbf{Latitude} & \textbf{Longitude}\\\hline
01544017        & \makecell{PEDRAS DE MARIA\\ DA CRUZ}     & JANUÁRIA           & -15,5978          & -44,3903 \\\hline
01544032        & \makecell{USINA DO PANDEIROS\\ MONTANTE} & JANUÁRIA           & -15,4831          & -44,7672 \\\hline
01544036        & \makecell{LONTRA}                        & LONTRA             & -15,9056          & -44,3072 \\\hline
%\multicolumn{5}{|c|}{}\\\hline
%\multicolumn{5}{|c|}{\textbf{Telemétricas}}\\\hline
%\multicolumn{5}{|c|}{\textbf{Fluviométricas}}\\\hline
%\textbf{Código}& \textbf{Nome}& \textbf{Município}& \textbf{Latitude}& \textbf{Longitude}\\\hline
%44252000 & \makecell{USINA DO PANDEIROS\\ JUSANTE} & JANUÁRIA & -15,5136 & -44,7537 \\\hline
\end{tabular}
\label{tab:estacoes_rio_sao_francisco}
\end{table}

É importante destacar algumas observações sobre as estações do rio Grande. Durante o período pesquisado, apenas foram encontrados dados de precipitação e vazão em estações localizadas no estado de São Paulo. As estações utilizadas para o rio Grande, as mais próximas da foz do rio e próximas à divisa com o estado de Minas Gerais, são aquelas listadas na tabela.

Uma situação semelhante ocorreu com o rio Doce. Não foram encontradas estações com dados disponíveis na foz do rio Doce, localizada no estado de Minas Gerais. Portanto, foi necessário utilizar a estação 56994500, situada no estado do Espírito Santo.

Estas são as únicas observações relevantes sobre as estações utilizadas.

\section{Pré-processamento dos Dados}
% Descrever os passos tomados para preparar os dados para análise (imputação de dados faltantes, gráficos de sazonalidade, etc...).

Com os dados disponíveis localmente, o primeiro passo antes de qualquer análise foi garantir a continuidade temporal dos mesmos. Existiam dias faltantes, e, para garantir uma linha do tempo contínua, foi necessário preencher essas lacunas. Os 11 anos de dados diários resultaram em um total de 4017 linhas de dados após essa etapa.

A sazonalidade é um fenômeno bem conhecido e estabelecido na análise hidrológica das bacias hidrográficas da América do Sul. O aumento da precipitação começa na primavera, em setembro, e atinge seus picos nos meses de dezembro e janeiro, durante o verão. Consequentemente, as vazões dos rios aumentam. Com a chegada do outono e, posteriormente, do inverno, os índices pluviométricos diminuem, assim como as vazões nos rios. \cite{rayyan-39677094}

Considerando esse fenômeno, o preenchimento dos dados faltantes foi realizado replicando o padrão sazonal. Para preencher um dia faltante em julho, por exemplo, foi utilizado o valor correspondente ao mesmo dia nos anos anteriores. Para evitar a repetição exata do ano anterior, utilizou-se a média dos últimos três anos. As funções desenvolvidas para essa finalidade são personalizáveis, permitindo que se opte por repetir exatamente o ano anterior ou considerar mais de três anos, dependendo das necessidades do estudo.

Note que a estratégia de realizar a média, para o dia, dos anos anteriores nem sempre preenchia exatamente as lacunas. Quando havia muitos dados faltantes no início da série isso causava problema e a inserção de dados falhava. O que é o comportamento normal.

Foi então que realizou-se uma nova contagem dos dados que ainda permaneciam faltantes. Para esses casos nulos, foi aplicada a imputação de dados utilizando o modelo kNN (k-Nearest Neighbors - k-vizinhos mais próximos), com o objetivo de garantir uma melhor dispersão dos valores imputados. O modelo kNN operou calculando a distância euclidiana dos pontos nulos utilizando os sete vizinhos mais próximos, atribuindo maior peso aos vizinhos mais próximos no cálculo. Esse método de imputação visou preservar a tendência local e o comportamento da série temporal dentro da semana em que o dado faltante estava. Após esta nova fase de imputação dos dados as séries ficaram completamente preenchidas.

É muito importante o destaque para esta fase de preenchimento de dados faltantes, e os desafios que isso apresentou ao trabalho, porque a escassez de informação foi um problema. Quando o período faltante era curto, o comportamento da série temporal preservou coerentemente os padrões sazonais, de tendência e estacionariedade. Contudo, mais especificamente para o rio Grande, isso tudo ainda não foi suficiente. A série temporal de vazão não preservou o comportamento sazonal esperado, ficando com muitos ruídos. Isso será mostrado adiante.

Neste momento cabe explicar uma nomenclatura utilizada no trabalho para rapidamente identificar o tipo de estação, se convencional ou telemétrica, de que dado ela trata (chuva ou vazão) e o código da estação. Tomemos dois exemplos que serão vistos nesta seção. Esta é a estação `c\_cv\_01941010', utilizada na análise do rio Doce. A letra `c' designa `convencional' e as letras `cv' significam `chuva', consequentemente, a sequência numérica é o código da estação registrado nos sistemas da ANA. A mesma analogia serve para as estações telemétricas. O nome `t\_cv\_54790000' significa `estação telemétrica de precipitação, código 54790000'.

\subsection{Rio Jequitinhonha}

A estação de vazão utilizada no rio Jequitinhonha apresentou uma quantidade significativa de dados faltantes, especialmente no início da série temporal. Observa-se uma clara sazonalidade na série, com picos de vazão ocorrendo predominantemente no final e início de cada ano (figura \ref{fig:jequitinhonhaSerieIncompleta_t_vz_54790000}). Nas páginas seguintes, serão apresentados gráficos comparativos entre a série original, sem dados imputados (incompleta), e a série após a imputação de dados (completa). Essa comparação, contrapondo a série fornecida pela ANA e os resultados após a inserção de dados, permitirá uma visualização mais clara do impacto das técnicas de preenchimento. Cabe relembrar que, inicialmente, foi aplicada a média dos últimos três anos para replicar a sazonalidade. Para os dados que permaneceram ausentes, utilizou-se o modelo kNN para completar a série. Esta estação, identificada como t\_vz\_54790000, apresentou 532 dias de dados nulos, o que corresponde a aproximadamente 13,24\% do total. Essa é a estação-alvo para a previsão das vazões.

\begin{figure}[!h]
\centering
\includegraphics[scale=0.25]{Figuras/jequiti/jequitinhonhaSerieIncompleta_t_vz_54790000.png}
\caption{Série temporal incompleta da estação t\_vz\_54790000 (fonte: o autor)}
\label{fig:jequitinhonhaSerieIncompleta_t_vz_54790000}
\end{figure}

A seguir, destaca-se o trecho da série com a maior quantidade de dados faltantes (figura \ref{fig:jequitinhonhaSerieIncompleta_t_vz_54790000-2013_2016}), que abrange o período de janeiro de 2013 a janeiro de 2016. Em sequência, é apresentada a série após a imputação dos dados (figura \ref{fig:jequitinhonhaSerieCompleta_t_vz_54790000-2013_2016}). Notavelmente, essa seção não apresentou resultados ideais, uma vez que a imputação atribuiu vazões zero em vários dias, o que não é realista, pois isso indicaria a secagem completa do rio, o que é improvável. No entanto, esses valores zero não impactaram significativamente os resultados finais da análise, já que se referem a um período distante do foco principal deste estudo. Uma alternativa seria excluir todo o trecho anterior ao ano de 2016, mas optou-se por manter a uniformidade nos critérios de aproveitamento dos dados ao longo do trabalho, dado que outros rios também foram analisados, e buscava-se assegurar consistência nos resultados.

\begin{figure}[!h]
\centering
\includegraphics[scale=0.25]{Figuras/jequiti/jequitinhonhaSerieIncompleta_t_vz_54790000-2013_2016.png}
\caption{Detalhe da série temporal da estação t\_vz\_54790000, ainda sem dados imputados, de 2013 a 2016 (fonte: o autor)}
\label{fig:jequitinhonhaSerieIncompleta_t_vz_54790000-2013_2016}
\end{figure}

\begin{figure}[!h]
\centering
\includegraphics[scale=0.25]{Figuras/jequiti/jequitinhonhaSerieCompleta_t_vz_54790000-2013_2016.png}
\caption{Detalhe da série temporal da estação t\_vz\_54790000, com dados imputados, de 2013 a 2016 (fonte: o autor)}
\label{fig:jequitinhonhaSerieCompleta_t_vz_54790000-2013_2016}
\end{figure}

Observe também o trecho de dados faltantes mais próximo ao final dos anos analisados, em 2021 e 2022 (figura \ref{fig:jequitinhonhaSerieIncompleta_t_vz_54790000-2021_2022}). Esta porção da série ficou boa visto que havia informação prévia suficiente, a inserção de dados respeitou coerentemente a sazonalidade (figura \ref{fig:jequitinhonhaSerieCompleta_t_vz_54790000-2021_2022}).

\begin{figure}[!h]
\centering
\includegraphics[scale=0.25]{Figuras/jequiti/jequitinhonhaSerieIncompleta_t_vz_54790000-2021_2022.png}
\caption{Série temporal incompleta da estação t\_vz\_54790000 no detalhe entre 2021 e 2022 (fonte: o autor)}
\label{fig:jequitinhonhaSerieIncompleta_t_vz_54790000-2021_2022}
\end{figure}

\begin{figure}[!h]
\centering
\includegraphics[scale=0.25]{Figuras/jequiti/jequitinhonhaSerieCompleta_t_vz_54790000-2021_2022.png}
\caption{Série temporal completa da estação t\_vz\_54790000 no detalhe entre 2021 e 2022 (fonte: o autor)}
\label{fig:jequitinhonhaSerieCompleta_t_vz_54790000-2021_2022}
\end{figure}

Por fim, uma visão ampla de como ficou a série temporal após os procedimentos de imputar os dados. (figura \ref{fig:jequitinhonhaSerieCompleta_t_vz_54790000})

\begin{figure}[!h]
\centering
\includegraphics[scale=0.25]{Figuras/jequiti/jequitinhonhaSerieCompleta_t_vz_54790000.png}
\caption{Série temporal completa da estação t\_vz\_54790000 (fonte: o autor)}
\label{fig:jequitinhonhaSerieCompleta_t_vz_54790000}
\end{figure}

A mesma análise foi realizada para as estações de chuva. Na estação t\_cv\_54790000 (figura \ref{fig:jequitinhonhaSerieIncompleta_t_cv_54790000}) faltavam 273 dias de dados (6,79\%). Já a estação t\_cv\_01640000 estava totalmente preenchida, sem valores nulos. (figura \ref{fig:jequitinhonhaSerieCompleta_t_cv_01640000})

\begin{figure}[!h]
\centering
\includegraphics[scale=0.25]{Figuras/jequiti/jequitinhonhaSerieIncompleta_t_cv_54790000.png}
\caption{Série temporal incompleta da estação t\_cv\_54790000 (fonte: o autor)}
\label{fig:jequitinhonhaSerieIncompleta_t_cv_54790000}
\end{figure}

Note que no início desta série de precipitação, o ano de 2013, não possuem dados. As séries de chuva completas ficaram desta forma (figuras \ref{fig:jequitinhonhaSerieCompleta_t_cv_54790000} e \ref{fig:jequitinhonhaSerieCompleta_t_cv_01640000})

\begin{figure}[!h]
\centering
\includegraphics[scale=0.25]{Figuras/jequiti/jequitinhonhaSerieCompleta_t_cv_54790000.png}
\caption{Série temporal completa da estação t\_cv\_54790000 (fonte: o autor)}
\label{fig:jequitinhonhaSerieCompleta_t_cv_54790000}
\end{figure}

\begin{figure}[!h]
\centering
\includegraphics[scale=0.25]{Figuras/jequiti/jequitinhonhaSerieCompleta_t_cv_01640000.png}
\caption{Série temporal completa da estação t\_cv\_01640000 (fonte: o autor)}
\label{fig:jequitinhonhaSerieCompleta_t_cv_01640000}
\end{figure}

\subsection{Rio Doce}

A estação alvo para o rio Doce é a estação c\_vz\_56994500. Sua série temporal foi a que apresentou melhor qualidade no que diz respeito à frequência de medições realizadas. Havia falta de apenas 3 dias, dos 4017 dias do período inteiro. Apenas o preenchimento sazonal bastou para completar a série e não foi preciso mais que isso. Cabe destacar a sazonalidade da série. Ficou bastante evidente este comportamento. (figura \ref{fig:doceSerieCompleta_c_vz_56994500})

\begin{figure}[!h]
\centering
\includegraphics[scale=0.25]{Figuras/rio_doce/doceSerieCompleta_c_vz_56994500.png}
\caption{Série temporal completa da estação c\_vz\_56994500 (fonte: o autor)}
\label{fig:doceSerieCompleta_c_vz_56994500}
\end{figure}

Se para os dados de vazão no rio Doce a série foi, digamos, mais comportada, o mesmo não se pode dizer exatamente das estações de chuva. Ao menos, não para duas delas. Estas estações tiveram os dados desconsiderados e foram removidos das análises. Primeiro foi a estação t\_cv\_56990850 que possuía valores discrepantes demais para serem considerados. Valores da ordem de 7000 mm/dia, 8500 mm/dia. Além deste problema, havia ainda 3134 dias com dados nulos, o que representava 78\% do total. (figura \ref{fig:doceSerieIncompleta_t_cv_56990850})

A outra estação removida foi a t\_cv\_56994500. Conforme pode ser observado na figura \ref{fig:doceSerieCompleta_t_cv_56994500}, nela havia um longo hiato de dados zerados, voltando à normalidade apenas mais recentemente. Como as informações de precipitação que deveria haver para a estação no período do hiato, pode ser retirado de outras estações usadas na modelagem, optou-se por remover esta estação completamente do trabalho.

\begin{figure}[!h]
\centering
\includegraphics[scale=0.25]{Figuras/rio_doce/doceSerieIncompleta_t_cv_56990850.png}
\caption{Série temporal da estação t\_cv\_56990850 - não utilizada (fonte: o autor)}
\label{fig:doceSerieIncompleta_t_cv_56990850}
\end{figure}

\begin{figure}[!h]
\centering
\includegraphics[scale=0.25]{Figuras/rio_doce/doceSerieCompleta_t_cv_56994500.png}
\caption{Série temporal da estação t\_cv\_56994500 - não utilizada (fonte: o autor)}
\label{fig:doceSerieCompleta_t_cv_56994500}
\end{figure}

As estações que, enfim, foram empregadas na modelagem são as que estão na tabela e, adiante, o gráfico da série temporal de cada uma delas. (figuras \ref{fig:doceSerieCompleta_c_cv_01941010}, \ref{fig:doceSerieCompleta_c_cv_01941004}, \ref{fig:doceSerieCompleta_c_cv_01941006} e \ref{fig:doceSerieCompleta_t_cv_56990005})

\begin{table}[h!]
\centering \small
\caption{Estações de precipitação usadas - final}
\begin{tabular}{|c|c|c|} \hline
\textbf{Estação} & \textbf{\# dados faltantes} & \textbf{\% dados faltantes} \\ \hline
c\_cv\_01941010  & 153                         & 3,81 \\ \hline
c\_cv\_01941004  & 31                          & 0,77 \\ \hline
c\_cv\_01941006  & 0                           & 0,00 \\ \hline
t\_cv\_56990005  & 1395                        & 34,73 \\ \hline
\end{tabular}
\label{tab:estacoes_chuva_usadas_final_rio_doce}
\end{table}

\begin{figure}[!h]
\centering
\includegraphics[scale=0.25]{Figuras/rio_doce/doceSerieCompleta_c_cv_01941010.png}
\caption{Série temporal completa da estação c\_cv\_01941010 (fonte: o autor)}
\label{fig:doceSerieCompleta_c_cv_01941010}
\end{figure}

\begin{figure}[!h]
\centering
\includegraphics[scale=0.25]{Figuras/rio_doce/doceSerieCompleta_c_cv_01941004.png}
\caption{Série temporal completa da estação c\_cv\_01941004 (fonte: o autor)}
\label{fig:doceSerieCompleta_c_cv_01941004}
\end{figure}

\begin{figure}[!h]
\centering
\includegraphics[scale=0.25]{Figuras/rio_doce/doceSerieCompleta_c_cv_01941006.png}
\caption{Série temporal completa da estação c\_cv\_01941006 (fonte: o autor)}
\label{fig:doceSerieCompleta_c_cv_01941006}
\end{figure}

\begin{figure}[!h]
\centering
\includegraphics[scale=0.25]{Figuras/rio_doce/doceSerieCompleta_t_cv_56990005.png}
\caption{Série temporal completa da estação t\_cv\_56990005 (fonte: o autor)}
\label{fig:doceSerieCompleta_t_cv_56990005}
\end{figure}

\subsection{Rio Grande}

O rio Grande apresentou desafios significativos ao longo de todo o desenvolvimento deste trabalho. A dificuldade inicial surgiu na ausência de dados disponíveis em estações dentro do estado de Minas Gerais para o período de análise estipulado, conforme mencionado anteriormente. Foi necessário buscar uma estação o mais próxima possível da divisa com Minas Gerais, localizada no estado de São Paulo, especificamente no município de Ilha Solteira. Entretanto, os desafios não se limitaram a essa questão geográfica.

A série temporal de vazão da estação selecionada, denominada t\_vz\_62020080, estava incompleta e não abrangia todo o período de 11 anos estipulado para a análise. (figura \ref{fig:grandeSerieIncompleta_t_vz_62020080}) Os dados disponíveis mais antigos datavam de 2020. Contudo, em conformidade com o escopo estabelecido para este estudo, foi realizado o preenchimento dos dados faltantes, aplicando-se o mesmo protocolo utilizado para os demais rios analisados. Este procedimento foi necessário para garantir a consistência, integridade e comparabilidade das análises subsequentes.

\begin{figure}[!h]
\centering
\includegraphics[scale=0.25]{Figuras/rio_grande/grandeSerieIncompleta_t_vz_62020080.png}
\caption{Série temporal incompleta da estação t\_vz\_62020080 (fonte: o autor)}
\label{fig:grandeSerieIncompleta_t_vz_62020080}
\end{figure}

Infelizmente, o caráter ruidoso da série permaneceu mesmo após a aplicação do protocolo de preenchimento dos dados ausentes, conforme pode ser observado na imagem final gerada. (figura \ref{fig:grandeSerieCompleta_t_vz_62020080}) A série em questão apresentava 2099 dias faltantes, correspondendo a aproximadamente 64\% de dados nulos. Outro aspecto relevante para essa estação é que, diferentemente das outras, não foram utilizados os 4.017 registros previstos inicialmente. As informações mais antigas disponíveis datavam de 2015, resultando, assim, em um total de 3289 registros diários utilizados especificamente para o rio Grande.

\begin{figure}[!h]
\centering
\includegraphics[scale=0.25]{Figuras/rio_grande/grandeSerieCompleta_t_vz_62020080.png}
\caption{Série temporal completa da estação t\_vz\_62020080 (fonte: o autor)}
\label{fig:grandeSerieCompleta_t_vz_62020080}
\end{figure}

A estação de precipitação utilizada, a única neste caso, foi a estação t\_cv\_61998080, pois foi a única que apresentou dados válidos. Curiosamente, outra estação de precipitação disponível também apresentou dados para o período analisado, mas a base de dados consistia exclusivamente em valores zero. Por essa razão, a estação t\_cv\_62020080 foi completamente excluída do estudo.

Em relação à estação t\_cv\_61998080, houve necessidade de preencher apenas um número reduzido de dados ausentes, totalizando 169 registros, o que correspondia a 5,14\% do total. (figura \ref{fig:grandeSerieCompleta_t_cv_61998080}) Trata-se de uma série com uma quantidade expressiva de dados, que efetivamente pôde contribuir de maneira significativa para as análises realizadas.

\begin{figure}[!h]
\centering
\includegraphics[scale=0.25]{Figuras/rio_grande/grandeSerieCompleta_t_cv_61998080.png}
\caption{Série completa da estação t\_cv\_61998080 (fonte: o autor)}
\label{fig:grandeSerieCompleta_t_cv_61998080}
\end{figure}

Ressalta-se que o trecho de dados faltantes para a estação t\_cv\_61998080 concentrava-se no início da série temporal, especificamente no ano de 2015. No gráfico os dados já estão imputados.

\subsection{Rio São Francisco}

Por fim, foi realizado o procedimento de preenchimento dos dados nulos para o Rio São Francisco. A estação-alvo c\_vz\_44290002 apresentou uma série bastante completa ao longo do período de análise, com apenas 120 dias nulos em um total de 4017 dias. O trecho com dados faltantes pode ser observado em detalhe na figura (\ref{fig:franciscoSerieIncompleta_c_vz_44290002-detalhe}).

Para esta estação, o preenchimento sazonal foi suficiente para suprir as lacunas existentes, não sendo necessário aplicar procedimentos adicionais de imputação de dados. (figura \ref{fig:franciscoSerieCompleta_c_vz_44290002})

\begin{figure}[!h]
\centering
\includegraphics[scale=0.25]{Figuras/rio_sao_francisco/franciscoSerieIncompleta_c_vz_44290002-detalhe.png}
\caption{Detalhe do trecho com dados nulos da estação c\_vz\_44290002 (fonte: o autor)}
\label{fig:franciscoSerieIncompleta_c_vz_44290002-detalhe}
\end{figure}

\begin{figure}[!h]
\centering
\includegraphics[scale=0.25]{Figuras/rio_sao_francisco/franciscoSerieCompleta_c_vz_44290002.png}
\caption{Série temporal completa da estação c\_vz\_44290002 (fonte: o autor)}
\label{fig:franciscoSerieCompleta_c_vz_44290002}
\end{figure}

No que se refere às estações de precipitação selecionadas para a análise no rio São Francisco, não foi necessário realizar nenhuma inserção de dados, uma vez que todas as séries estavam completas, abrangendo a totalidade dos 4017 dias de registro. As séries temporais correspondentes podem ser visualizadas nos gráficos apresentados a seguir. (figuras \ref{fig:franciscoSerieCompleta_c_cv_01544017}, \ref{fig:franciscoSerieCompleta_c_cv_01544032}, \ref{fig:franciscoSerieCompleta_c_cv_01544036})

\begin{figure}[!h]
\centering
\includegraphics[scale=0.25]{Figuras/rio_sao_francisco/franciscoSerieCompleta_c_cv_01544017.png}
\caption{Série temporal completa da estação c\_cv\_01544017 (fonte: o autor)}
\label{fig:franciscoSerieCompleta_c_cv_01544017}
\end{figure}

\begin{figure}[!h]
\centering
\includegraphics[scale=0.25]{Figuras/rio_sao_francisco/franciscoSerieCompleta_c_cv_01544032.png}
\caption{Série temporal completa da estação c\_cv\_01544032 (fonte: o autor)}
\label{fig:franciscoSerieCompleta_c_cv_01544032}
\end{figure}

\begin{figure}[!h]
\centering
\includegraphics[scale=0.25]{Figuras/rio_sao_francisco/franciscoSerieCompleta_c_cv_01544036.png}
\caption{Série temporal completa da estação c\_cv\_01544036 (fonte: o autor)}
\label{fig:franciscoSerieCompleta_c_cv_01544036}
\end{figure}


\section{Modelos de Aprendizado de Máquina}
%Apresentar os modelos de ML utilizados (SeasonalNaive, LinearRegression, CatBoost e LightGBM) e justificativas.

\subsection{Seasonal Naive}

O modelo \textbf{Seasonal Naive} não pode ser considerado um modelo de previsão sofisticado. Em vez disso, ele funciona como uma linha de base (baseline), servindo como ponto de partida para avaliar o desempenho de outros modelos de previsão. Este modelo simplesmente repete a sazonalidade observada no período anterior, ou seja, assume que o comportamento do próximo ciclo sazonal será o mesmo do anterior, com um possível ajuste por \textit{drift} (desvio).

Esse método é útil para fornecer uma idéia inicial do comportamento esperado, permitindo que os modelos subsequentes sejam comparados a ele. Por ser um modelo simples, ele não captura tendências ou variações complexas, mas estabelece um \textit{benchmark} mínimo para o qual outros métodos mais elaborados e complexos devem se comparar.

\subsection{Linear Regression}

O modelo de Regressão Linear (\textit{Linear Regression}) é uma abordagem estatística simples, porém poderosa, que busca modelar o relacionamento entre uma variável dependente e uma ou mais variáveis independentes através de uma linha reta. 

O funcionamento da Regressão Linear envolve o cálculo de coeficientes para as variáveis independentes, que determinam o peso de cada uma destas variáveis na previsão da variável dependente. O objetivo do modelo é minimizar o erro quadrático médio, ou seja, a soma dos quadrados das diferenças entre os valores previstos e os valores reais.

Os resultados obtidos com este modelo mostraram-se muito bons, e na verdade, ele se destacou como um modelo de base robusto para os outros modelos mais complexos. Sua simplicidade e eficácia tornam-no uma escolha bastante sólida. Isso será discutido.

Aqui está o fluxograma para o funcionamento do modelo de Regressão Linear

\todo[inline]{INSERIR DIAGRAMA ?}

\subsection{CatBoost}

CatBoost é um algoritmo de aprendizado de máquina baseado em árvores de decisão que se destaca pela sua eficiência e capacidade de lidar com variáveis categóricas sem necessidade de pré-processamento extensivo. É possível especificar as variáveis categóricas já no momento de criação do objeto, de acordo com a implementação mais usual do modelo.

Seu nome é uma abreviação de ``\textit{Categorical Boosting}'' e o mesmo é um tipo de modelo de \textit{gradient boosting} que cria sequencialmente árvores de decisão (modelos fracos), onde cada árvore é ajustada para corrigir os erros cometidos pelas árvores anteriores. Abaixo está uma explicação passo-a-passo de como o CatBoost funciona:

\begin{itemize}
\item Preprocessamento de Variáveis Categóricas: CatBoost pode tratar variáveis categóricas de forma automática, utilizando uma codificação baseada em permutação em que as categorias são transformadas em números levando em consideração a dependência da ordem.
\item Construção de Árvores: O modelo cria uma série de árvores de decisão em sequência. Cada árvore tenta corrigir os erros das árvores anteriores. Elas, as árvores, são construídas utilizando uma amostra aleatória dos dados o que ajuda a reduzir o viés.
\item Gradient Boosting: O CatBoost utiliza um método de \textit{gradient boosting}, onde o modelo tenta minimizar uma função de perda ajustando os pesos das observações e focando nos erros cometidos nas iterações anteriores.
\item Regularização: Para evitar \textit{overfitting}, o CatBoost incorpora técnicas de regularização, como a introdução de uma pequena aleatoriedade no processo de construção das árvores e a suavização das previsões para variáveis categóricas.
\item Predição Final: As previsões finais são uma média ponderada das previsões de todas as árvores, com cada árvore contribuindo com base na sua capacidade de reduzir o erro.
\end{itemize}

Esse modelo se destaca pela sua capacidade de manejar dados com muitas variáveis categóricas, sem perda de desempenho. Pode ser uma ótima escolha para conjuntos de dados grandes com relações complexas, como, exatamente, dados de precipitação e vazão.

\begin{figure}[!h]
\centering
\includegraphics[scale=0.75]{Figuras/catboost_diagrama.jpg}
\caption{Diagrama resumindo o funcionamento do CatBoost (fonte: o autor)}
\label{fig:catboost_diagrama}
\end{figure}

\subsection{LightGBM}

O \textbf{LightGBM} também é um modelo baseado em \textit{gradient boosting} e em termos de comportamento é semelhante em funcionamento ao CatBoost. Contudo, o processo de treinamento aplica uma técnica denominada ``Leaf-wise growth''. Esta estratégia, ao invés do tradicional \textit{Level-wise Growth}, o crescimento ocorre folha a folha, a árvore expande as folhas com a maior redução de erro, resultando em árvores mais profundas e precisas, porém também mais suscetíveis ao \textit{overfitting}, o que é controlado por hiperparâmetros como por exemplo a profundidade máxima (\textit{max-depth}) da árvore.

As principais características do LightGBM incluem:
- Alta eficiência em memória e velocidade, permitindo treinar modelos em grandes \textit{datasets};
- Capacidade de lidar com elevado número de \textit{features};
- Ajuste de alta precisão mesmo em cenários com dados desbalanceados.

Fluxograma do funcionamento do LightGBM:
\todo[inline]{INSERIR DIAGRAMA ?}

%A escolha do modelo certo para um problema de previsão depende das características dos dados e dos objetivos do estudo. O SeasonalNaive, embora simples, é um importante ponto de referência inicial. O modelo de Regressão Linear serve como uma baseline confiável devido à sua eficácia e simplicidade. Já os modelos CatBoost e LightGBM são opções mais avançadas, capazes de lidar com a complexidade dos dados e oferecer previsões precisas e eficientes. A comparação entre esses modelos permite que se escolha a abordagem que melhor atenda às necessidades específicas da previsão, garantindo assim resultados mais robustos e precisos.

%\section{Variáveis Utilizadas}
%Listar e explicar as variáveis contínuas e categóricas utilizadas nos modelos.

%\section{Treinamento e Validação dos Modelos}
%Detalhar os procedimentos de treinamento e validação dos modelos.

\section{Modelo Proposto}
% Detalhar os procedimentos de treinamento e validação dos modelos.
\begin{figure}[!h]
\centering
\includegraphics[scale=0.3]{Figuras/flowchart.png}
\caption{Fluxo de trabalho (fonte: o autor)}
\label{fig:fluxo_trabalho}
\end{figure}

%\section{Métricas de Avaliação}
%Listar as métricas usadas para avaliar o desempenho dos modelos (MAPE, RMSE, PBias* e KGE não-paramétrico**).
%(*: PBias ajuda a ver se o modelo está subestimando as previsões ou superestimando)
%(**: KGEnp (não paramétrico) é uma métrica comum e muito consagrada na hidrologia. Ela tem resiliência a outliers ao avaliar a qualidade do modelo)